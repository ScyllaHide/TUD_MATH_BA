%TODO fix references ... :/
\begin{enumerate}[label=]
	\item \ul{Ziel:} Für $X_1, X_2, \dots$ Zufallsvariablen (u.i.v) betrachten das Langzeitmittel $Y:= \frac{1}{n} \sum_{i=1}^n X_i$ für $n \to \infty$.
	\item  \ul{Wissen bereits:} Für $X_i \sim \Ber(p), p \in (0,1)$ unabhängig
	\begin{itemize}
		\item Für $np \to \lambda, n \to \infty$:
		\begin{align*}
			\P(n Y_n = k) \xrightarrow{n \to \infty} e^{-\lambda}\frac{\lambda^k}{k!} = \Pois(\set{k}) 
G		\end{align*}
		(\person{Poisson}-Approx. \propref{2_6}).
		\item Für $c > 0$ gilt mit $x(k) := \frac{k-np}{\sqrt{np(1-p)}}$
		\begin{align*}
			\lim_{n\to \infty} \max_{k: \abs{x_n(k)} \le c} \abs{\frac{\sqrt{np(1-p)}\P(nY_n = k)}{g(x_n(k))} -1} = 0
		\end{align*}
		(\person{De Moivre}-\person{Laplace} \propref{7_2}).
	\end{itemize}
\end{enumerate}
\section{Schwaches Gesetz der großen Zahlen}
(WLLN- Weak law of large numbers)\\
Sei $X_1, X_2, \dots$ u.i.v. $\Ber(\frac{1}{2})$ Zufallsvariablen (z.B. Münzwurf). Wir erwarten, etwa bei der Hälfte der Zufallsvariablen eine Null/Eins (Kopf/Zahl) zu sehen, also
\begin{align*}
	Y_n = \frac{1}{n}\sum_{i=1}^n X_i \xrightarrow{n \to \infty} \frac{1}{2} = \E[X_n] \quad \text{im geeigneten Sinne.}
\end{align*}
Tatsächlich folgt mit der \person{Stirling}-Formel (\propref{7.3)}
\begin{align*} %TODO add cancel package?
	\P(Y_{2n} = \frac{1}{2}) &= \P(\Bin(2n,p) = \frac{2n}{2})\\
	&= \binom{2n}{n} 2^{-2n}\\
	\over{n \to \infty}&{\sim} \frac{\sqrt{2\pi 2n}(2n)^{2n}e^{-2n}}{(\sqrt{2 \pi n}n^n e^{-n})^2} 2^{-2n}\\
	&= \frac{1}{\sqrt{\pi n}}\\
	&\xrightarrow{n \to \infty} 0
\end{align*}
Wir benötigen also einen geeigneten Grenzwertbegriff!
\begin{definition}[\begriff{stochastische Konvergenz}]
	Seien $Y, Y_1, Y_2, \dots$ reelle Zufallsvariablen auf einem Wahrscheinlihckeitsraum $(\O,\F, \P)$. Falls für alle $\epsilon > 0$ gilt
	\begin{align*}
		\lim_{n\to \infty} \P(\abs{Y_n - Y} \le \epsilon) = 1
	\end{align*}
	so konvergiert $(Y_n)_{n \in \N}$ \begriff{stochastisch} oder \begriff{in Wahrscheinlichkeit} gegen $Y$.\\
	%TODO flip stuff in xrightarrow
	Schreibweise: $Y_n \xrightarrow[\P]{n \to \infty} Y$ oder $\P-\lim_{n\to \infty} Y_n = Y$. (oder auch $\plim_{n \to \infty} Y_n =Y$).
\end{definition}
\begin{*remark}
	Für Zufallsvariablen in $\Rd$ gilt eine entsprechende Definition mit euklidischer Norm anstelle des Betrags.
\end{*remark}
\begin{lemma}
	Seien $Y,Z,Y_1, Y_2, \dots$ Zufallsvariablen auf einem Wahrscheinlichkeitsraum $(\O,\F,\P)$. Gelten
	\begin{align*}
		Y_n \xrightarrow[\P]{n \to \infty} Y \und Y_n \xrightarrow[\P]{n \to \infty} Z
		\intertext{so folgt}
		Y=Z \quad \P-\text{f.s.}
	\end{align*}
\end{lemma}
\begin{proof}
	$\Delta$-Ungleichung: $\abs{Y-Z} \le \abs{Y-Y_n} + \abs{Y_n-Z}$. Damit
	\begin{align*}
		\set{\abs{Y-Z} > 2 \epsilon} \le \set{\abs{Y-Y_n} > \epsilon} \cup \set{\abs{Y_n-Z} > \epsilon}
		\intertext{so dass}
		\P(\abs{Y-Z} > 2\epsilon) \le \P(\abs{Y-Y_n} > \epsilon) + \P(\abs{Y_n-Z} > \epsilon) \xrightarrow{n \to \infty} 0
		\end{align*}
		Also
		\begin{align*}
		\P(Y \neq Z) = \P(\abs{Y-Z} > 0) &= \P(\bigcup_{k=1}^{\infty} \set{\abs{Y-Z} > k^{-1}})\\
		&= \sum_{k=1}^{\infty} \P(\abs{Y-Z} > k^{-1}) = 0 \implies Y=Z \quad \P-\text{f.s.}
	\end{align*}
\end{proof}
\begin{lemma}
	Seien $Y_1,Y_2, \dots, Z_1, Z_2, \dots$ reelle Zufallsvariablen auf $(\O, \F,\P)$ und $(a_n)_{n\in \N}$ reellwertige Folge. Es gelten:
	\begin{enumerate}
		\item $Y_n \xrightarrow[\P]{n \to \infty} 0$ und $Z_n \xrightarrow[\P]{n \to \infty} 0$, dann $Y_n + Z_n \xrightarrow[\P]{n \to \infty} 0$.
		\item $Y_n \xrightarrow[n \to \infty]{\P} 0$ und $(a_n)_{n\in \N}$ beschränkt, dann
		\begin{align*}
			a_n Y_n \xrightarrow[n \to \infty]{\P} 0.
		\end{align*}
	\end{enumerate}
\end{lemma}
\begin{proof}
	Für $\epsilon > 0$ beliebig gilt
	\begin{align*}
		\P(\abs{Y_n-Z_n} > \epsilon) \le \P(\abs{Y_n} > \frac{\epsilon}{2}) + \P(\abs{Z_n} > \frac{\epsilon}{2}) \xrightarrow[n \to \infty]{\P} 0.
	\end{align*}
	und andererseits, wenn $\abs{a_n} \le A$
	\begin{align*}
		\P(\abs{a_n Y_n}> \epsilon) \le \P(\abs{Y_n} > \frac{\epsilon}{A}) \xrightarrow[n ßto \infty]{\P} 0.
	\end{align*}
\end{proof}
\begin{proposition}[WLLN, $\Ln{2}$-Version]
	$(\O,\F,\P)$ Wahrscheinlichkeitsraum, $X_1, X_2, \dots$ paarweise unkorrelierte, reelle Zufallsvariablen auf $\O$ in $\Ln{2}(\P)$, so dass
	\begin{align*}
		v := \sup_{i \in \N} \Var(X_i) < \infty
	\end{align*}
	Dann gilt für alle $\epsilon > 0$.
	\begin{align*}
		\P(\abs{1/n \sum_{i=1}^n (X_i - \E[X_i])} \ge \epsilon) \le \frac{v}{n\epsilon^2}
		\intertext{also insbesondere}
		1/n \sum_{i=1}^n (X_i - \E[X_i]) \xrightarrow[n \to \infty]{\P} 0
	\end{align*}
	Falls $\E[X_i] = \E[X_1] \quad \forall i$, so gilt
	\begin{align*}
		1/n \sum_{i=1}^n X_i \xrightarrow[n \to \infty]{\P} \E[X_1]
	\end{align*}
\end{proposition}
\begin{proof}
	Sei $Y_n := 1/n \sum_{i=1}^n (X_i - \E[X_i])$, dann ist $Y_n \in \Ln{2}(\P)$ mit $\E[Y_n] = 0$ und
	\begin{align*}
		\Var(Y_n) = \frac{1}{n} \sum_{i=1}^n \Var(X_i) \le \frac{V}{n} \quad \text{\person{Bienamé}} 
	\end{align*}
	und  die Aussage folgt aus der Ungleichung von \person{Tschebyscheff} (\propref{5_13}).
\end{proof}
\begin{proposition}[WLLN, $\Ln{1}$-Version]
	$(\; \F, \P)$ Wahrscheinlichkeitsraum, $X_1, X_2, \dots$ paarweise unabhängige, identisch verteilte reelle Zufallsvariablen auf $\O$ in $\Ln{1}(\P)$. Dann gilt
	\begin{align*}
		\frac{1}{n} \sum_{i=1}^n X_i \xrightarrow[n \to \infty]{\P} \E[X_1].
	\end{align*}
\end{proposition}
\begin{proof}
	Wir verwenden ein Abschneideargument:\\
	Definiere
	\begin{align*} %TODO find a way for good {} size, also later ... if not manual maybe?!
		X_i^2 := X_i \indi_{\set{\abs{X_i} \le i^{\frac{1}{4}}}}
		\intertext{und}
		X_i^{\sharp} := X_i - X_i^{b} = X_i \indi_{\set{\abs{X_i} \le i^{\frac{1}{4}}}}
		\intertext{sowie}
		Y_n^{\flat} := \frac{1}{n} \sum_{i=1}^n (X_i^{\flat} - \E[X_i^{\flat}])\\
		Y_n^{\sharp} := \frac{1}{n} \sum_{i=1}^n \sum_{i=1}^n (X_i^{\sharp} - \E[X_i^{\sharp}]) 
	\end{align*}
	Wir zeigen $Y_n^{flat} \xrightarrow[n \to \infty]{\P} 0$ und $Y_n^{\sharp} \xrightarrow[n \to \infty]{\P} 0$.
	Mit \propref{9_3} folgt dann die Behauptung. Zu $Y_n^{\flat}$: Nach \propref{3_19} sind auch $X_i^{\flat}$ paarweise unabhängig. Es folgt
	\begin{align*}
		\Var(Y_n^{flat}) &= \frac{1}{n^2} \sum_{i=1}^n \Var(X_i^{\flat}) \quad \text{(\person{Bienayé})}\\
		&\le \frac{1}{n^2} \sum_{i=1}^n \E[\underbrace{(X_i^{\flat})^2}_{\le i^{\frac{1}{2}}}]\\
		&\le \frac{1}{n^2}n n^{\frac{1}{2}} = n^{-\frac{1}{2}}
	\end{align*}
	Mit \person{Tschebyscheff} folgt $Y_n^{\flat} \xrightarrow[n \to \infty]{\P} 0$.\\
	Zu $Y_n^{\sharp}$: Es gilt
	\begin{align*}
		\E[X_i^{\sharp}] &= \E[X_i \indi_{\set{\abs{X_i} \le i^{\frac{1}{4}}}}] \over{\text{i.v.}}{=} \E[X_i \indi_{\set{\abs{X_i} \le i^{\frac{1}{4}}}}]\\
		&= \E[X_1] - \E[X_1 \indi_{\set{\abs{X_i} \le i^{\frac{1}{4}}}}]\\
		& \xrightarrow[n \to \infty]{} 0.
	\end{align*}
	wegen monotoner Konvergenz. Also
	\begin{align*}
		\E[\abs{Y_n^{\sharp}}] \le \frac{2}{n} \sum_{i=1}^n \E[X_i^{\sharp}] \xrightarrow[n \to \infty]{} 0.
	\end{align*}
	Mit \person{Markov}-Ungleichung (\propref{5_6}) folgt für alle $\epsilon > 0$
	\begin{align*}
		\P(\abs{Y_n^{\sharp}} \ge \epsilon) \le \frac{\E[\abs{Y_n^{\sharp}}]}{\epsilon} \xrightarrow[n \to \infty]{} 0
	\end{align*}
	also $Y_n^{\sharp} \xrightarrow[n \to \infty]{\P} 0$. 
\end{proof}