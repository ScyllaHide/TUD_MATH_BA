\section{$\Ln{p}$-Konvergenz}
\begin{definition}[$\Ln{p}$-Konvergenz]
	Seien $Y,Y_1,Y_2, \dots$ reelle Zufallsvariablen auf $(\O,\F\,\P)$. Falls $(Y,Y_n \in \Ln{p}(\P)), n\in \N$ für ein $p \in [1,\infty]$ und
	\begin{align*}
		\lim_{n \to \infty} \norm{Y-Y_n}_p := \lim_{n \to \infty} 
		(\E[Y_n - Y]^p)^{\sfrac{1}{p}} = 0
	\end{align*}
	so \begriff{konvergiert} $(Y_n)_{n \in \N}$ in $\Ln{p}$/ im $p$-ten Mittel gegen $Y$. Wir schreiben:
	\begin{align*}
		Y_n \xrightarrow[n \to \infty]{\Ln{p}} Y \oder \Ln{p}-\lim_{n \to \infty} Y_n = Y
	\end{align*}
\end{definition}
\begin{*remark}
	\begin{itemize}
		\item Der Grenzwert einer $\Ln{p}$-konvergenten Folge ist f.s. eindeutig: \person{Minkowski}-Ungleichung ($\nearrow$ Schilling MINT, Korollar 14.5) liefert für $1 \le p \le \infty$
		\begin{align*}
			\norm{Y-Z}_p = \norm{Y-Y_n+Y_n - Z}_p \le \underbrace{\norm{Y-Y_n}_p}_{\to 0} + 
			\underbrace{\norm{Y_n-Z}_p}_{\to 0}\\
			\implies \norm{Y-Z}_p = \implies Y=Z \quad \P\text{-f.s.}
		\end{align*}
		\item $\Ln{p}$-Konvergenz in $\Rd$ lässt sich analog definieren.
	\end{itemize}
\end{*remark}
\begin{lemma}
	$Y,Y_1,Y_2,\dots$ reelle Zufallsvariablen auf $(\O,\F,\P)$. Es gelten für $1 \le p \le \infty$
	\begin{enumerate}
		\item $Y_n \xrightarrow[n \to \infty]{\Ln{p}} Y \implies Y_n \xrightarrow[n \to \infty]{\Ln{1}} Y$
		\item $Y_n \xrightarrow[n \to \infty]{\Ln{1}} Y\implies Y_n \xrightarrow[n \to \infty]{\P} Y$
	\end{enumerate}
\end{lemma}
\begin{proof}
	\begin{enumerate}
		\item Für $1 \le p \le \infty$ setze $q = (1- 1/p)^{-1}$ mit $q = 1$ für $p = \infty$ so dass $1/p + 1/q = 1$ und mit \person{Hölder}
		\begin{align*}
			\norm{Y_n-Z}_1 &= \E[Y_n-Y] = \int \abs{Y_n-Y}\cdot 1 \d \P\\
			&= \brackets{\int \abs{Y_n -Y}^p}^{1/p} \cdot \underbrace{\brackets{\int 1^q \d \P}^{1/q}}_{=1}\\
			&= \brackets{\E[\abs{Y_n -Y}^p]}^{1/p}\\
			&= \abs{Y_n - Y}_p \xrightarrow[n \to \infty] 0
		\end{align*}
		\item Mit der \person{Markov}-Ungleichung folgt $\forall \epsilon > 0$
		\begin{align*}
			\P(\abs{Y_n - Y} > \epsilon) \le \frac{\E[\abs{Y_n - Y}]}{\epsilon} \xrightarrow[n \to \infty]{} 0.
		\end{align*}
	\end{enumerate}
\end{proof}
\begin{center}
	\begin{tikzpicture}
	\node at (0,0) (a) {f.s. Konv.};
	\node at (4,0) (b) {$\mathscr{L}^p$-Konv.};
	\node at (0,-2) (c) {Stoch. Konv.};
	\node at (4,-2) (d) {$\mathscr{L}^1$-Konv.};
	\node at (0,-4) (e) {Verteilungskonv.};
	
	\draw[->, dotted, -triangle 45] (b) to node [above] {\small TF} (a);
	\draw[->, double, -triangle 45] (b) -- (d);
	\draw[->, -triangle 45, double] (a) to node [right] {$\uparrow$ \small TF} (c);
	\draw[->, -triangle 45, double] (d) -- (c);
	\draw[->, -triangle 45, double] (c) -- (e);
	\end{tikzpicture}
\end{center}