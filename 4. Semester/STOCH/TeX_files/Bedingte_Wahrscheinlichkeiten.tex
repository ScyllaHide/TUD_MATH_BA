\section{Bedingte Wahrscheinlichkeiten}
\begin{example}
	\proplbl{3_1}
	Das Würfeln mit zwei fairen, sechsseitigen Würfeln können wir mit 
	\begin{align}
		\O = \set{(i,j) \colon i,j \in \set{1,\dots,6}}\notag
	\end{align}
	und $\P = \Gleich(\O)$. Da $\abs{\O} = 36$ gilt also
	\begin{align}
		\P(\set{\omega}) = \frac{1}{36} \quad \forall \omega \in \O.\notag
	\end{align}
	Betrachte das Ereignis
	\begin{align}
		A = \set{(i,j) \in \O \colon i + j = 8},\notag
	\end{align}
	dann folgt
	\begin{align}
		\P(A) = \frac{5}{36}.\notag
	\end{align}
	Werden die beiden Würfe nacheinander ausgeführt, so kann nach dem ersten Wurf eine Neubewertung der Wahrscheinlichkeit von $A$ erfolgen.\\
	Ist z.B.
	\begin{align}
		B = \set{(i,j) \in \O, i = 4}\notag
	\end{align}
	eingetreten, so kann die Summe $8$ nur durch eine weitere $4$ realisiert werden, also mit Wahrscheinlichkeit
	\begin{align}
		\frac{1}{6} = \frac{\abs{A \cap B}}{\abs{B}}.\notag 
	\end{align}
	Das Eintreten von $B$ führt also dazu, dass das Wahrscheinlichkeitsmaß $\P$ durch ein neues Wahrscheinlichkeitsmaß $\P_{B}$ ersetzt werden muss. Hierbei sollte gelten:
	\begin{align}
		 &\text{Renormierung: }\P_{B}(\O) = 1\label{Renorm}\tag{R}\\
		 &\text{Proportionalität: Für alle} A \subseteq \F \mit A \subseteq B \text{ gilt }
		 \P_{B}(A) = c_B \P(A) \text{ mit einer Konstante } c_B.\label{Prop}\tag{P}
    \end{align}
\end{example}

\begin{lemma}
	\proplbl{3_2}
	Sei $(\O, \F, \P)$ Wahrscheinlichkeitsraum und $B \in \F$ mit $\P(B) > 0$. Dann gibt es genau ein Wahrscheinlichkeitsmaß $\P_B$ auf $(\O, \F)$ mit den Eigenschaften \eqref{Renorm} und \eqref{Prop}. Dieses ist gegeben durch
	\begin{align}
		\P_{B}(A) = \frac{\P(A\cap B)}{\P(B)} \quad \forall A \in \F.\notag
	\end{align}
\end{lemma}

\begin{proof}
	Offenbar erfüllt $\P_{B}$ wie definiert \eqref{Renorm} und \eqref{Prop}. Umgekehrt erfüllt $\P_{B}$ \eqref{Renorm} und \eqref{Prop}. Dann folgt für $A \in \F$:
	\begin{align}
		\P_{B}(A) = \P_{B}(A\cap B) + \underbrace{\P_{B}(A\setminus B)}_{= 0, \text{ wegen } \eqref{Renorm}} \overset{\eqref{Prop}}{=} c_B \P(A \cap B).\notag
	\end{align}
	Für $A=B$ folgt zudem aus \eqref{Renorm}
	\begin{align}
		1 = \P_{B}(B) = c_B \P(B)\notag
	\end{align}
	also $c_B = \P(B)^{-1}$.
\end{proof}

% % % % % % % % % % % % % % % % % % % % % % % % % % % 5th lecture % % % % % % % % % % % % % % % % % % % % % % % % % % %

\begin{definition}[Bedingte Wahrscheinlichkeit]
	\proplbl{3_3}
	Sei $(\O, \F, \P)$ Wahrscheinlichkeitsraum und $B \in \F$ mit $\P(B) > 0$. Dann heißt
	\begin{align*}
		\P(A\mid B) := \frac{\P(A\cap B)}{\P(B)} \mit A\in \F
	\end{align*}
	die \begriff{bedingte Wahrscheinlichkeit von $A$ gegeben $B$}.
	Falls $\P(B) = 0$, setze
	\begin{align*}
		\P(A \mid B) = 0 \qquad \forall A \in \F
	\end{align*}
\end{definition}

\begin{example} %TODO ref
	\proplbl{3_4}
	In der Situation \propref{3_1} gilt % 
	\begin{align*}
		A \cap B = \set{(4,4)}
		\intertext{und damit}
		\P(A \mid B) = \frac{\P(A\cap B)}{\P(B)} = \frac{\frac{1}{36}}{\frac{1}{6}} = \frac{1}{6}
	\end{align*}
\end{example}

Aus \propref{3_3} ergibt sich
\begin{lemma}[Multiplikationsformel]
	\proplbl{3_5}
	Sei $(\O, \F, \P)$ ein Wahrscheinlichkeitsraum und $A_1, \dots, A_n \in \F$. Dann gilt
	\begin{align*}
		\P(A_1 \cap \cdots \cap A_n) = \P(A_1) \P(A_2 \mid A_1) \dots \P(A_n \mid A_1 \cap \cdots \cap A_{n-1})
	\end{align*}
\end{lemma}

\begin{proof}
	Ist $\P(A_1 \cap \dots \cap A_n) = 0$, so gilt auch $\P(A_n \mid \bigcap_{i=1}^{n-1} A_i) = 0$. Andernfalls sind alle Faktoren der rechten Seite ungleich Null und
	\begin{align*}
		\P(A_1) \P(A_2 \mid A_1) \dots \P(A_n \mid \bigcap_{i=1}^{n-1} A_i)
		&= \P(A_1) \cdot \frac{\P(A_1 \cap A_2)}{\P(A_1)} \dots \frac{\P(\bigcap_{i=1}^{n} A_i)}{\P(\bigcap_{i=1}^{n-1}A_i)} \\
		&= \P(\bigcap_{i=1}^n A_i)	
	\end{align*}
\end{proof} %TODO add ref.

Stehen die $A_i$ in \propref{3_5} in einer (zeitlichen) Abfolge, so liefert Formel einen Hinweis wie Wahrscheinlichkeitsmaße für \begriff{Stufenexperimente} konstruiert werden können. Ein \emph{Stufenexperiment} aus $n$ nacheinander ausgeführten Teilexperimenten lässt sich als \begriff{Baumdiagramm} darstellen.

%(done) Baumdiagramm
\begin{center}
		\includegraphics{./tikz/baum_1.pdf}
		\captionof{figure}{\cref{3_5}} %\propref{3_1_4}} funktioniert nicht
\end{center}

\begin{proposition}[Konstruktion des Wahrscheinlichkeitsmaßes eines Stufenexperiments]
	\proplbl{3_6}
	Gegeben seinen $n$ Ergebnisräume $\O_i = \set{\omega_i (1), \dots, \omega_i (k)}, k \in \N \cup \set{\infty}$ und es sei $\O = \bigtimes_{i = 1}^n \O_i$ der zugehörige Produktraum. Weiter seinen $\F_i$ $\sigma$-Algebren auf $\O_i$ und $\F = \bigotimes_{i=1}^n \F_i$ die Produkt-$\sigma$-Algebra auf $\O$. Setze $\omega = (\omega_1,\dots,\omega_n)$ und
	\begin{align*}
		[\omega_1,\dots,\omega_m]:= \set{\omega_1}\times \dots \times \set{\omega_m} \times \O_{m+1} \times \cdots \times \O_{n},\quad m\le n\\
		\P(\set{\omega_m}[\omega_1,\dots,\omega_{m-1}])
	\end{align*}
	für die Wahrscheinlichkeit in der $m$-ten Stufe des Experiments $\omega_m$ zu beobachten, falls in den vorausgehenden Stufen $\omega_1,\dots,\omega_{m-1}$ beobachten wurden. Dann definiert
	\begin{align*}
		\P(\set{\omega}) := \P(\set{\omega_1}) \prod_{m=2}^{n}\P\brackets{\set{\omega_m} \mid [\omega_1, \dots, \omega_{m-1}]}
		%(done) maybe wrong here. check
	\end{align*}
	ein Wahrscheinlichkeitsmaß auf $(\O, \F, \P)$.
\end{proposition}
\begin{proof}
	Nachrechnen!
\end{proof}

\begin{example}[\person{Polya}-Urne]
	\proplbl{3_7}
	Gegeben sei eine Urne mit $s$ schwarzen und $w$ weißen Kugeln. Bei jedem Zug wird die  gezogene Kugel zusammen mit $c\in \N_0 \cup \set{-1}$ weiteren Kugeln derselben Farbe zurückgelegt.
	\begin{itemize} %TODO seen both in chapter 2.2, but big bracket behind.
		\item $c=0$: Urnenmodell mit Zurücklegen
		\item $c=-1$: Urnenmodell ohne Zurücklegen
	\end{itemize}
	Beide haben wir schon in Kapitel 2.2 gesehen.\\
	Sei deshalb $c\in \N$. (Modell für zwei konkurrierende Populationen) Ziehen wir $n$-mal, so haben wir ein $n$-Stufenexperiment mit 
	\begin{align*}
		\O = \set{0,1}^n \mit \text{ 0 = ``weiß'', 1 = ``schwarz''} \quad (\O_i = \set{0,1})
		\intertext{Zudem gelten im ersten Schritt}
		\P(\set{0}) = \frac{w}{s+w} \und \P(\set{1}) = \frac{s}{s+w}
		\intertext{sowie}
		\P(\set{\omega_m} \mid [\omega_1, \dots \omega_{m-1}]) = 
		\begin{cases} %(done) fix brackets!
		\frac{w+c \brackets{m-1 - \sum_{i=1}^{m-1}\omega_i}}{s+w+c(m-1)} & \omega_m = 0\\
		\frac{s + c\sum_{i=1}^{m-1}\omega_i}{s+w+c(m-1)} & \omega_m = 1
		\end{cases}
	\end{align*}
	Mit \propref{3_6} folgt als Wahrscheinlichkeitsmaß auf $(\O, \pows(\O))$
	\begin{align*}
		\P(\set{(\omega_1, \dots, \omega_n)}) &= \P(\set{\omega_1}) \prod_{m=2}^n \P(\set{\omega_m}\mid [\omega_1,\dots,\omega_{m-1}]) \\
		&=\frac{\prod_{i=0}^{l-1}(s+c \cdot i)\prod_{i=0}^{n-l-1}(w + c \cdot j)}{\prod_{i=0}^n (s+w+c \cdot i)} \mit l=\sum_{i=1}^n \omega_i.
		\intertext{Definiere wir nun die Zufallsvariable}
		S_n:\O &\to \N_0 \mit (\omega_1, \dots, \omega_n) \mapsto \sum_{i=1}^n \omega_i
		\intertext{welche die Anzahl der gezogenen schwarzen Kugeln modelliert, so folgt}
		\P(S_n = l) &= \binom{n}{l} \frac{\prod_{i=0}^{l-1}(s+c \cdot i) \prod_{j=0}^{n-l-1}(w + c \cdot j)}{\prod_{i=0}^n(s+w+c \cdot i)}
		\intertext{Mittels $a:= \sfrac{s}{c},b:= \sfrac{w}{c}$ folgt}
		\P(S_n = l) &= \binom{n}{l} \frac{\prod_{i=0}^{l-1}(-a-i)\prod_{j=0}^{n-l-1}(-b-j)}{\prod_{i=0}^n (-a-b-i)} = \frac{\binom{-a}{l}\binom{-b}{n \cdot l}}{\binom{-a-b}{n}}\\ &\mit l \in \set{0,\dots,n} 
	\end{align*}
	Dies ist die \begriff{\person{Polya}-Verteilung} auf $\set{0,\dots,n}, n \in \N$ mit Parametern $a,b > 0$.
\end{example}

\begin{example}
	\proplbl{3_8}
	Ein Student beantwortet eine Multiple-Choice-Frage mit 4 Antwortmöglichkeiten, eine davon ist richtig. Er kennt die richtige Antwort mit Wahrscheinlichkeit $\sfrac{2}{3}$. Wenn er diese kennt, so wählt er diese aus. Andernfalls wählt er zufällig (gleichverteilt) eine Antwort. Betrachte
	\begin{align*}
		W &= \set{\text{richtige Antwort gewusst}}\\
		R &= \set{\text{Richtige Antwort gewählt}}
		\intertext{Dann gilt}
		\P(W) &= \frac{2}{3}, \P(R \mid W) = 1, \P(R \mid W^C) = \frac{1}{4} 
	\end{align*}
	Angenommen, der Student gibt die richtige Antwort. Mit welcher Wahrscheinlichkeit hat er diese gewusst? $\longrightarrow \P(W\mid R) = \text{ ?}$
\end{example}

\begin{proposition}
	\proplbl{3_9}
	Sei $(\O, \F, \P)$ Wahrscheinlichkeitsraum und $\O = \bigcup_{i \in I} B_i$ eine höchstens abzählbare Zerlegung in paarweise disjunkte Ereignisse $B_i \in \F$.
	\begin{enumerate} %TODO set itemize references. or use enumerate?
		\item \emph{Satz von der totalen Wahrscheinlichkeit:} Für alle $A \in \F$ gilt
		\begin{align*}
			\P(A) = \sum_{i\in I} \P(A\mid B_i)\P(B_i) \label{eq:totWkeit}\tag{totale Wahrscheinlichkeit}
		\end{align*} 
		\item \emph{Satz von \person{Bayes}:} Für alle $A \in \F$ mit $\P(A) > 0$ und alle $k \in I$
		\begin{align*}
			\P(B_k \mid A) = \frac{\P(A \mid B_k) \P(B_k)}{\sum_{i\in I}\P(A\mid B_i)\P(B_i)} \label{eq:bayes}\tag{Bayes}
		\end{align*}
	\end{enumerate}
\end{proposition}

\begin{proof}
	\begin{enumerate}
		\item Es gilt:
		\begin{align*}
			\sum_{i\in I} \P(A\mid B_i)\P(B_i) \defeq \sum_{i\in I}\frac{\P(A \cap B_i)}{\P(B_i)}\P(B_i) = \sum_{i\in I} \P(A \cap B_i) \overset{\sigma-Add.}{=} \P(A)
		\end{align*}
		\item 
		\begin{align*}
			\P(B_k \mid A) \defeq \frac{\P(A \cap B_k)}{\P(A)} \defeq \frac{\P(A \mid B_k)\P(B_k)}{\P(A)}
		\end{align*}
		also folgt (b) aus (a). %TODO add refs
	\end{enumerate}
\end{proof}

\begin{example}
	\proplbl{3_10}
	In der Situation von \propref{3_3} folgt mit \propref{3_9} \eqref{eq:totWkeit}
	\begin{align*}
		\P(R) &= \P(R \mid W)\P(W) + \P(R\mid W^C)\P(W^C)\\
		&= 1 \cdot \frac{2}{3} + \frac{1}{4} \frac{1}{3} = \frac{3}{4}
		\intertext{und mit \propref{3_9} \eqref{eq:bayes}} %Bayes
		\P(W \mid R) &= \frac{\P(R \mid W)\P(W)}{\P(R)} = \frac{1 \cdot \frac{2}{3}}{\frac{3}{4}} = \frac{8}{9} \text{ für die gesuchte Wahrscheinlichkeit.}
	\end{align*} %(done) compile as pdf and include it. not working.
	\begin{center}
			\includegraphics{./tikz/baum_2.pdf}
			\captionof{figure}{\cref{3_3}} %\propref{3_3} funktioniert nicht
	\end{center}
\end{example}