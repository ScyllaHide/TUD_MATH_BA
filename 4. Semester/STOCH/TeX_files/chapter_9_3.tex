\section{Der Satz von \person{Glivenko}-\person{Cantelli}}
Gegeben seien $n$ Realisierungen $x_1 = X_1(\omega), x_2 = X_2(\omega), \dots x_n = X_n(\omega)$ von u.i.v. Zufallsvariablen $X_1,\dots, X_n$ in $\R$ auf einem Wahrscheinlichkeitsraum $(\O, \F, \P)$.\\
Die Verteilung $F$ von $X_i$ ist unbekannt.
\begin{enumerate}[label=]
	\item \ul{Frage:} Wie können wir $\P_{x_i}$ bzw. die zugehörige Verteilungsfunktion $F$ bestimmen/ approximieren?
\end{enumerate}
\begin{definition}
	Seien $X_1,X_2, \dots$ u.i.v. Zufallsvariablen auf $(\O,\F,\P)$ in $\R$, dann definiert
	\begin{align*}
		\hat{F}_n(x,\omega) := \frac{\# \set{ i\colon 1 \le i \le n \mid X_i(\omega) \le x}}{n}
	\end{align*}
	die \begriff{empirische Verteilungsfunktion} der \begriff{Zufallsproben} $(X_1, \dots, X_n)$.
\end{definition}
\begin{*remark}
	\begin{itemize}
		\item Für alle $\omega \in \O$ ist $\hat{F}_n(x): \R \to [0,1]$ eine Verteilungsfunktion ($\nearrow$ Übung)
		\item $\set{\hat{F}_n, x \in \R}$ ist eine Familie von Zufallsvariablen und damit ein \begriff{stochasticher Prozess}.
	\end{itemize}
\end{*remark}
\begin{proposition}[Eigenschaften der empirischen Verteilungsfunktion]
	\proplbl{9_13} Seien $X_1, X_2, \dots$ u.i.v. Zufallsvariablen auf $(\O,\F,\P)$ in $\R$ mit Verteilungsfunktion $F$. Dann gelten:
	\begin{enumerate}
		\item $n \hat{F}_n(x) \sim \Bin(n,F(x)) \quad \forall x\in \R$
		\item $\E[\hat{F}_n(x)] = F(x) \quad \forall x \in \R$
		\item $\Var(\hat{F}_n(x)) = \frac{F(x)(1-F(x))}{n} \quad \forall x \in \R$
		\item $\hat{F}_n(x) \xrightarrow[n \to \infty]{\text{f.s.}} F(x) \quad x \in \R$
	\end{enumerate}
\end{proposition}
\begin{proof} %fix formating!
	\begin{enumerate}
		\item Es gilt
		\begin{align*}
			\hat{F}_n(x) = \frac{1}{n} \sum_{i=1}^n \indi_{\set{X_i \le x}}
			\intertext{mit}
			\indi_{\set{X_i \le x}} \sim \Ber(F(x)) % bernoulli
		\end{align*}
		Mit Unabhängigkeit der $X_i$ folgt die Behauptung.
		\item (2\&3) folgt aus 1) $\forall x$ 
		\begin{align*}
			\E[n\hat{F}_n(x)] = nF(x)\\
			\Var(n \hat{F}_n(x))(1-F(x))
		\end{align*}
		\item Da die Zufallsvariablen $Y_i = \indi_{\set{X_i \le x}}$ für jedes $x$ u.i.v. und in $\Ln{2}$ sind, gilt nach dem SLLN
		\begin{align*}
			\hat{F}_(x) &= \frac{1}{n} Y_i \xrightarrow[n \to \infty]{\text{f.s.}} \E[Y_i]\\
			&= \E[\indi_{\set{X_i \le x}}] = \P(X_i \le x) = F(x).
		\end{align*}
	\end{enumerate}
\end{proof}
\begin{proposition}[\person{Gilvenko}-\person{Cantelli}]
	\proplbl{9_14} $X_1,X_2, \dots$ u.i.v. Zufallsvariablen auf $(\O,\F,\P)$ in $\R$ mit Verteilungsfunktion $F$. Dann gilt
	\begin{align*}
		\P(\lim_{n\to \infty} \sup_{x \in \R} \abs{\hat{F}_n(x) - F(x)} = 0) = 1\\
		\sup_{x \in \R} \abs{\hat{F}_n(x) - F(x)} \xrightarrow[n \to \infty]{\text{f.s.}} 0.
	\end{align*}
\end{proposition}
\begin{proof}
	Nehme zunächst an, dass $F$ stetig ist. Für jedes $m \in \N$ existieren dann\begin{align*}
		-\infty = z_0 < z_1 < \cdots < z_{m-1} < z_m = \infty
		\intertext{so dass}
		F(z_0) = \quad F(z_1) = \frac{1}{m}\quad ... F(z_k) = \frac{k}{m} \quad ... F(z_n) = 1
	\end{align*}
	Es folgt für jedes $z \in [z_n, z_{k+1})$
	\begin{align*}
		\hat{F}_n(z) - F(z) \le \hat{F}_n(z_{k+1}) - F(z_k) = \hat{F}(z_{k+1})-F(z_{k+1}) + \frac{1}{m}\\
		\hat{F}_n(z) - F(z) \ge \hat{F}_n(z_n)-F(z_{k+1}) = \hat{F}_n(z_k) - F(z_k) - \frac{1}{m} \tag{$\star$} \label{bew:9_14:a}
	\end{align*}
	Setze für $k \in \set{0,1, \dots, m}, m \in \N$. Dann
	\begin{align*}
		A_{m,k} := \set{\omega \in \O \colon \lim_{n\to \infty} \hat{F}_n(z_k,\omega) = F(z_n)}
	\end{align*}
	dann gilt nach \propref{9_13} $\P(A_{m,k}) = 1\quad \forall m,k$. Für
	\begin{align*}
		A_m := \bigcup_{k=0}^m A_{m,k}
		\intertext{folgt}
		\P(A_m) = 1 - \P(A_m^C) = 1-\P(\bigcup A^C_m) \ge 1 - \sum \P(A^C_{m,k}) = 1
	\end{align*}
	Für jedes $\omega \in A_m$ gilbt es $N(\omega)\in \N$, so dass $\forall n \ge N(\omega)$ und $\forall k \in \set{0, \dots, m}$
	\begin{align*}
		\abs{\hat{F}_n(z_k, \omega) - F(z_k)} < \frac{1}{m}
	\end{align*}
	Zusammen mit \eqref{bew:9_14:a} folgt $\forall n \ge N(\omega)$ und alle $\omega \in A_m$
	\begin{align*}
		\sup_{x \in \R} \abs{\hat{F}_n(z) - F(z)} < \frac{2}{m} \tag{$\star\star$} \label{bew:9_14:b}
	\end{align*}
	Für $A = \bigcap_{m \in \N}A_m$ folgt (wie bei $A_m$) auch $\P(A) = 1$ und aus \eqref{bew:9_14:b}, dass für jedes $\omega \in A$ und jedes $\epsilon > 0$ ein $N(\omega,\epsilon) \in \N$ existiert, so dass $\forall n > N(\omega, \epsilon)$ gilt
	\begin{align*}
		\sup_{z \in \R} \abs{\hat{F}_n(z, \omega) - F(z)} < \epsilon
	\end{align*} 
	Für $\epsilon \to 0$ folgt die Behauptung.\\
	Ist $F$ nicht notwendigerweise steig, so wähle für $m \in \N$
	\begin{align*}
		- \infty = z_0 < z_1 < \cdots < z_{m-1} < z_m = \infty
		\intertext{so dass}
		F(z_{k+1}^-) - F(z_k) = \lim_{y\uparrow \infty} F(y) - F(z_k) \le \frac{1}{m}
	\end{align*}
	Dann folgt analog zu  \eqref{bew:9_14:a} für $z \in [z_n, z_{k+1})$
	\begin{align*}
		\hat{F}_n(z) - F(z) \le \hat{F}_(z_{k+1}^-) - F(z_{k+1}^-) + \frac{1}{m}\\
		\hat{F}\hat{F}_n(z) -F(z) \ge \hat{F}_n(z_n) - F(z_k) - \frac{1}{m}
		\intertext{und damit}
		A'_{m,k} = \set{\omega \in \O, \lim_{n\to \infty} \hat{F}_n(z_k^-,\omega) = F(z_k^-)}
	\end{align*}
	folgt der verbleibende Beweis analog zum stetigen Fall.
\end{proof}