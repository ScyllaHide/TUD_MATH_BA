\documentclass[ngerman,a4paper,order=firstname]{../../texmf/tex/latex/mathscript/mathscript}
\usepackage{../../texmf/tex/latex/mathoperators/mathoperators}

% % % local commands
\newcommand{\Halb}{\mathfrak{X}}            % Halbordnung
\newcommand{\caly}{\mathcal{Y}}				% caligraphic Y
\DeclareMathOperator{\transdeg}{tr.deg}		% transcendence degree
\DeclareMathOperator{\Tr}{Tr}				% Trace Matrix or here for L\mid K
\newcommand{\Zwischen}{\mathscr M}

\newlist{remarkenum}{enumerate}{1}
\setlist[remarkenum]{label=(\alph*),ref=\theremark~(\alph*)}
\crefalias{remarkenumi}{remark}

\newlist{propenum}{enumerate}{1}
\setlist[propenum]{label=(\alph*),ref=\theproposition~(\alph*)}
\crefalias{propenumi}{proposition}

\newlist{expenum}{enumerate}{1}
\setlist[expenum]{label=(\alph*),ref=\theexample~(\alph*)}
\crefalias{expenumi}{example}

\title{\textbf{Algebra und Zahlentheorie SS 2019}}
\author{Dozent: Prof. Dr. \person{Arno Fehm}}

\begin{document}
\pagenumbering{roman}
\pagestyle{plain}

\maketitle

\hypertarget{tocpage}{}
\tableofcontents
\bookmark[dest=tocpage,level=1]{Inhaltsverzeichnis}

\pagebreak
\pagenumbering{arabic}
\pagestyle{fancy}

\chapter*{Vorwort}
\input{./TeX_files/Vorwort}
\chapter*{Motivation und Einführung}
\input{./TeX_files/Motivation}
\chapter{Körper}
\input{./TeX_files/Korper/Korpererweiterungen}
\include{./TeX_files/Korper/Alg_Korpererweiterungen}
\section{Wurzelkörper und Zerfällungskörper}
Sei $K$ ein Körper, $f \in K[X]$ mit $n = \deg(f) > 0$.
\begin{example}
	Sei $K=\Q$. Dann hat $f$ eine Nullstelle (``Wurzel'') $\alpha \in \C$, und $L:= K(\alpha) = K[\alpha]$ ist die kleinste Erweiterung von $\Q$ in $\C$, die diese Nullstelle enthält.
\end{example}
\begin{definition}[Wurzelkörper]
	Ein \begriff{Wurzelkörper} von $f$ ist eine Körpererweiterung $L \mid K$ der Form $L = K(\alpha)$ mit $f(\alpha) = 0$.
\end{definition}
\begin{lemma}
	\proplbl{1_3_3}
	Sei $L = K(\alpha)$ mit $f(\alpha) = 0$ ein Wurzelkörper von $f$. Dann ist $[L:K] \le n$. Ist $f$ irreduzibel, so ist $[L:K] = n$ und $g \mapsto g(\alpha)$ induziert einen Isomorphismus $\lnkset{K[X]}{(f)} \overset{\cong}{\longrightarrow}_K L$.
\end{lemma}
\begin{proof} %TODO fix b) ref
	Sei zunächst $f$ irreduzibel, $f_{\alpha} = \MinPol(\alpha \mid K)$. Dann ist $f = cf_{\alpha}$, die Behauptung folgt somit aus \propref{korpererweiterungen:prop:1:2:7:b}. Für $f \in K[X]$ beliebig, schreibe $f = f_1\cdots f_r$ mit $f_i \in K[X]$ irreduzibel und
	\begin{flalign*}
		\qquad &f(\alpha) = 0 \quad\Rightarrow\quad \text{O.E. } f_1(\alpha) = 0 \quad\Rightarrow\quad [L:K] = \deg(f_1) \le \deg(f) = n& %TODO is it really 1 in the index?
	\end{flalign*}
\end{proof}
\begin{lemma}
	\proplbl{1_3_4}
	Sei $f$ irreduzibel. Dann ist $L := \lnkset{K[X]}{(f)}$ ein Wurzelkörper von $f$.
\end{lemma}
\begin{proof}
	Betrachte den Epimorphismus $\pi = \pi_f\colon K[X] \to \lnkset{K[X]}{(f)} = L$, setze $\alpha = \pi(X)$
	\begin{itemize}
		\item $K$ Körper $\Rightarrow \pi_{\mid K}$ injektiv\\
		$\Rightarrow$ können $K$ mit Teilkörper von $L$ identifizieren, sodass $\pi_{\mid K} = \id_K$
		\item $(f)$ irreduzibel $\Rightarrow$ prim $\xRightarrow{\text{GEO II.4.7}}$ $(f)$ maximal $\Rightarrow L = \lnkset{K[X]}{(f)}$ ist Körper
		\item $f(\alpha) = f(\pi(X)) \overset{(\ast)}{=} \pi(f(X)) = 0$ $\Rightarrow$ $f(X) \in \Ker(\pi)$\\
		($\ast$ gilt, da $f = \sum a_i x^i$ $\Rightarrow$ $\pi(f) = \sum \pi(a_i)\pi(x)^i = \sum a_i \pi(x)^i = f(\pi(x))$)
		\item $L=\pi(K[X]) = K[\pi(X)] = K[\alpha] \overset{\alpha \text{ alg.}}{=} K(\alpha)$
	\end{itemize}
\end{proof}
\begin{proposition}
	\proplbl{1_3_5}
	Sei $f$ irreduzibel. Ein Wurzelkörper von $f$ existiert und ist eindeutig in folgendem Sinn:\\
	Sind $L_1 = K(\alpha_1), L_2 = K(\alpha_2)$ mit $f(\alpha_1) = 0 = f(\alpha_2)$, so existiert genau ein $K$-Isomorphismus $\varphi\colon L_1 \to L_2$ mit $\varphi(\alpha_1) = \alpha_2$.
\end{proposition}
\begin{proof}\leavevmode\vspace*{\dimexpr-\baselineskip+2\lineskip}
	\begin{itemize}
		\item Existenz gibt \propref{1_3_4}
		\item \propref{1_3_3} liefert Isomorphismus
		\begin{flalign*}
			\qquad & \left.\begin{matrix} %TODO find a way to only have the right curly bracket!
				L_1 \xleftarrow[\varphi_1]{\cong} & \lnkset{K[X]}{(f)} & \xrightarrow[\varphi_2]{\cong} L_2\\
				\alpha_1 \mapsfrom & X + (f) & \mapsto \alpha_2\\
			\end{matrix}\right\rbrace
			\quad\Rightarrow \quad\varphi_2 \circ \varphi_1\colon L_1 \xrightarrow{\cong}_K L_2 \mit \alpha_1 \mapsto \alpha_2 &
		\end{flalign*}
		Umgekehrt ist jeder $K$-Isomorphismus $\varphi\colon L_1 \to_K L_2$ wegen $L_1 = K(\alpha_1)$ schon durch $\varphi(\alpha_1)$ festgelegt.
	\end{itemize}
\end{proof}
\begin{conclusion}
	\proplbl{1_3_6}
	$f$ hat einen Wurzelkörper.
\end{conclusion}
\begin{proof}
	Schreibe $f=f_1\cdots f_r, f_1,\dots,f_r \in K[X]$ irreduzibel, nehme einen Wurzelkörper von $f_1$.
\end{proof}
\begin{conclusion}
	\proplbl{1_3_7}
	Es gibt eine Erweiterung $L\mid K$, über der $f$ in Linearfaktoren zerfällt, also $f=c\prod_{i=0}^{n}(x-\alpha_i)$ mit $c \in K^{\times}$, $\alpha$, $\dots$, $\alpha_n \in L$. 
\end{conclusion}
\begin{proof}\NoEndMark
	Schreibe $f=c\cdot f_0 \mit c \in K^{\times}, f_0 \in K[X]$ normiert.\\ Induktion nach $n$:
	\vspace*{\dimexpr-\baselineskip+2\lineskip}
	\begin{description}[leftmargin=4em,labelindent=1em]
		\item[$n=1{:}$] $f = x-a$, nehme $L=K$.
		\item[$n>1{:}$] Nach \propref{1_3_6} existiert $L_1 \mid K$, $\alpha_1 \in L_1$ mit $f_0 (\alpha_1) = 0$\\
		\begin{tabularx}{\linewidth}{@{\hspace*{0.5em}}r@{$\;\;$}X}
		$\Rightarrow$ & $f_0 = (x-\alpha_1)\cdot f_1 \mit f_1 \in L_1 [X]$ normiert\\
		$\xRightarrow{\text{(IH)}}$ & Es existiert $L\mid L_1$, $\alpha_1$, $\dots$, $\alpha_n \in L$ mit $f_1 = \prod_{i=2}^n (x - \alpha_i)$\\
		$\Rightarrow$  & $f = c\cdot f_0 = c\cdot (x-\alpha_1) \cdot f_1 = c \prod_{i=1}^n (x- \alpha_i)$\hfill\csname\InTheoType Symbol\endcsname
		\end{tabularx}
	\end{description}
\end{proof}
\begin{definition}[Zerfällungskörper]
	Ein \begriff{Zerfällungskörper} von $K$ ist eine Erweiterung $L\mid K$ der Form $L = K(\alpha_1,\dots,\alpha_n)$ mit $f=c\mal \prod_{i=1}^n (x-\alpha_i)$ und $c \in K^{\times}$.
\end{definition}
\begin{proposition}
	\proplbl{1_3_9}
	Ein Zerfällungskörper von $f$ existiert.
\end{proposition}
\begin{proof}
	Ist $L\mid K$ wie in \propref{1_3_7}, ist $K(\alpha_1,\dots,\alpha_n)$ ein Zerfällungskörper von $f$.
\end{proof}
\begin{lemma}
	Ist $L \mid K$ ein Zerfällungskörper von $f$, so ist $[L:K] \le n$!
\end{lemma}
\begin{proof}\NoEndMark
	Sei $L = K(\alpha_1,\dots,\alpha_n)$, $f = c\prod_{i=1}^n (x-\alpha_i)$.\\
	Induktion nach $n$:
	\vspace*{\dimexpr-\baselineskip+3\lineskip}
	\begin{description}[leftmargin=4em,labelindent=1em]
		\item[$n=1{:}$] $L=K$, $[K:K] = 1$
		\item[$n>1{:}$] $L_1 = K(\alpha_1)$ ist Wurzelkörper von $f$
		
		\begin{tabularx}{\linewidth}{@{\hspace*{0.5em}}r@{$\;\;$}X}
			 $\xRightarrow{\propref{1_3_3}}$  &  $[L_1:K] \le n$ und schreibe $f=c\mal (x-\alpha_1)\mal f_1, f_1 = \prod_{i=2}^n (x-\alpha_i) \in L_1[X]$\\
			$\Rightarrow$ & $L = K(\alpha_1,\dots,\alpha_n) = L_1(\alpha_1,\dots,\alpha_n)$ ist Zerfällungskörper von $f_1$ (über $L_1$)\\
			$\xRightarrow{\text{IH}}$ & $[L:L_1] \le \deg(f_1)! = (n-1)!$\\
			$\Rightarrow$ & $[L:K] = [L:L_1][L_1:K] = (n-1)!\,n = n!$\hfill\csname\InTheoType Symbol\endcsname
		\end{tabularx}
	\end{description}
\end{proof}
\begin{example}
	\begin{expenum}
		\item Ist $n=2$, so ist jeder Wurzelkörper $L$ von $f$, schon ein Zerfällungskörper: $[L:K]\le 2$.
		\item \proplbl{1_3_11_b} Ist $n =3$, $f$ irreduzibel. Schreibe $L_1 = K(\alpha), f = c(x-\alpha_1)f_1 \mit f_1 \in L_1[X]$
			\begin{itemize}
				\item $f_1$ reduzibel: $L_1$ ist schon Zerfällungskörper von $f$, $[L_1:K] = 3$
				\item $f_1$ irreduzibel: $L_1$ ist kein Zerfällungskörper von $f$. Ist $L$ Wurzelkörper von $f_1$, so ist $L$ Zerfällungskörper von $f$, $[L:K] = 3! = 6$
			\end{itemize}
	\end{expenum}
\end{example}
\begin{*example}
	Sei $f = x^3 -2 \in \Q[X]$, dann sind die Nullstellen von $f$: $\sqrt[3]{2} \in \R$, $\zeta_3\sqrt[2]{2}$, $\zeta_3^2 \sqrt[2]{2}$
	\begin{itemize}
		\item $\Q(\sqrt[2]{2})$ ist Wurzelkörper von $f$. $\Q(\sqrt[3]{2}) \subseteq \R$, $\zeta_3\sqrt[3]{2}$, $\zeta_3^2 \sqrt[3]{2} \notin \R$, aber kein Zerfällungskörper. Der Zerfällungskörper von $f$ ist
		\begin{align*}
			\Q(\sqrt[3]{2},\zeta_3\sqrt[3]{2}, \zeta_3^2 \sqrt[3]{2}) = \Q(\sqrt[3]{2}, \zeta_3 \sqrt[3]{2})
		\end{align*}
	\end{itemize}
\end{*example}
\begin{mathematica}
	Will man die Nullstellen von $f = X^3 - 2 \in \Q[X]$ finden, dann bietet Mathematica folgende Funktion:
	\begin{align*}
		\texttt{Solve[f==0,x,Complexes]},
	\end{align*}
	der letzte Parameter lässt einem den Körper wählen, in dem Mathematica suchen soll. Es gibt zur Auswahl \texttt{Integers, Rationals, Reals, Complexes}. Für das Beispiel erhält man folgenden Output:
	\begin{align*}
		\set{x \to -(-2)^{(1/3)}, x \to 2^{(1/3)}, x \to (-1)^{(2/3)} 2^{(1/3)}}.
	\end{align*}
	Dabei müsste man die Einheitswurzeln identifizieren:
	\begin{align*}
		\set{x \to \zeta_3\sqrt[3]{2}, x \to \sqrt[3]{2}, x \to \zeta_3^2 \sqrt[3]{2}}
	\end{align*}
\end{mathematica}
\begin{*anmerkung}
	Wenn $f$ irreduzibel $\Rightarrow \lnkset{K[X]}{(f)}$ ist Wurzelkörper.
\end{*anmerkung}
\begin{lemma}
	\proplbl{1_3_12}
	Sei $f = \sum_{i=0}^n a_i X^i$ irreduzibel und sei $L = K(\alpha)$ mit $f(\alpha)=0$ ein Wurzelkörper von $f$. Sei $L'\mid K'$ eine weitere Körpererweiterung und $\varphi \in \Hom(K,K')$. Ist $\sigma \in \Hom(L,L')$ eine Fortsetzung von $\varphi$ (d.h. $\sigma_{\mid K} = \varphi$), so ist $\sigma(\alpha)$ eine Nullstelle von $f^{\varphi}=\sum_{i=0}^n \varphi(\alpha_i)X^i \in K[X]$.
	
	Ist umgekehrt $\beta \in L' $ eine Nullstelle von $f^{\varphi}$, so gibt es genau eine Fortsetzung $\sigma \in \Hom(L,\tilde{L})$ von $\varphi$ mit $\sigma(\alpha) = \beta$.
\begin{center} % tikzcd was bitchy, compiled and included the pdf.
	\includegraphics{./tikz/lemma_1_3_12.pdf}
\end{center}
\end{lemma}
\begin{proof}[was für die Prüfung!]\leavevmode\vspace*{\dimexpr-\baselineskip+2\lineskip}
	\begin{itemize}
		\item $f(\alpha) = 0$ $\Rightarrow$ $0 = \sigma(0) = \sigma\big(f(\alpha)\big) = \sigma\big(\sum_{i=0}^n a_i \alpha^i\big) = \sum_{i=0}^n \varphi(a_i)\sigma(\alpha)^i = f^{\varphi}\big(\sigma(\alpha)\big)$
		\item Eindeutigkeit klar, da $L=K(\alpha)$
		\item Existenz: Betrachte 
		\begin{align*}
			\eta\colon&
			\left\lbrace\begin{array}{@{}l@{\;}c@{\;}l}
				K[X] &\to& L\\
				g &\mapsto& g(\alpha)
			\end{array}\right.&
			\psi\colon&
			\left\lbrace\begin{array}{@{}l@{\;}c@{\;}l}
				K[X] &\to& L'\\
				g &\mapsto& g^{\varphi}(\beta) 
			\end{array}\right.
		\end{align*}
		Beide sind Homomorphismen nach der universellen Eigenschaft.
		(Bemerke: $\eta$ surjektiv: $\eta_{\mid K} = \id \to K \subset \Image(\eta)$ mit $\eta(X) = \alpha \to \alpha \in \Image(\eta)$)\\
		Aus $\Ker(\eta)=(f)$ folgt der Isomorphismus $\bar{\eta}\colon \lnkset{K[X]}{(f)} \xrightarrow{\cong}L$ und\\
		$f \in \Ker(\psi) \Rightarrow \Ker(\psi) = (f)$ liefert Homomorphismus $\bar{\psi}\colon \lnkset{K[X]}{(f)} \to L'$\\
		$\sigma:= \bar{\psi}\circ \bar{\eta}^{-1}\colon L \to L'$ ist eine Fortsetzung von $\phi$ und
		\begin{align*}
			\sigma(\alpha) = \bar{\psi}(X+(f)) = \beta
		\end{align*}
	\end{itemize}
\end{proof}
\begin{proposition}
	\proplbl{1_3_13}
	Der Zerfällungskörper von $f$ ist eindeutig bestimmt bis auf $K$-Isomorphie.
\end{proposition}
\begin{proof} Für den Beweis betrachte erst folgende Aussage.
	\begin{adjustwidth}{1em}{-6pt}
	\begin{underlinedenvironment}[Behauptung]
		Ist $\varphi\colon K \to K'$ ein Isomorphismus, $L$ ein Zerfällungskörper und $L'$ ein Zerfällungskörper von $f^{\varphi}$, so setzt sich $\varphi$ zu einem Isomorphismus $L \to L'$ fort.
	\end{underlinedenvironment}
	\vspace*{-\baselineskip}
	\begin{proof}\NoEndMark
			Induktion nach $n = \deg(f)$:
			\vspace*{-4\lineskip}
			\begin{itemize}[leftmargin=4.5em,itemsep=-2\lineskip] %TODO maybe use enum here without label?
				\item[$n=1$:] $L = K \xrightarrow[\varphi]{\cong} K' = L'$ \checkmark
				\item[$n>1$:] Schreibe $f = cg_1\cdots g_r$ mit $g_i \in K[X]$ normiert und irreduzibel, $c \in K^{\times}$\\[-0.8mm]
				\begin{tabularx}{\linewidth}{@{\hspace{0.5em}}r@{$\;\;$}X}
					$\Rightarrow$ & $f^{\varphi} = c^{\varphi}g_1^{\varphi}\cdots g_r^{\varphi}$ mit $c^{\varphi}\in (K')^{\varphi}$ und $g_i^{\varphi}\in K' [X]$ normiert und irreduzibel (weil $\varphi$ Isomorphismus ist). Sei $\alpha_1 \in L$ mit $g_1 (\alpha_1) = 0$, $\alpha'_1 \in L'$ mit $g_1^{\varphi}(\alpha'_1) = 0$\\
					$\xRightarrow{\propref{1_3_12}}$ & \begin{minipage}[t]{\linewidth}
						$\varphi$ setzt man zu einem Isomorphismus
					\[
						\sigma\colon K_1 := K(\alpha_1) \to K' (\alpha'_1) \quad \mit \sigma(\alpha_1) = \alpha'_1
					\]
					fort.
					\end{minipage} \\[16mm]
					\multicolumn{2}{l}{Schreibe $f=(x - \alpha_1)\cdot f_1$ mit $f_1 \in K_1 [X]$} \\[-2mm]
					$\Rightarrow$ & $f^{\varphi} = \big(x - \underbrace{\sigma(\alpha_1)}_{\alpha'_1}\big)\cdot f_1^{\sigma}$ mit $f_1^{\sigma}\in K'_1 [X]$.\\[-2mm]
					&$L$ ist Zerfällungskörper von $f_1$, $L'$ ist Zerfällungskörper von $f_1^{\sigma}$\\
					$\Rightarrow$ &  $\sigma$ setzt sich fort zu einem Isomorphismus $L \to L'$\hfill\csname\InTheoType Symbol\endcsname
				\end{tabularx}
			\end{itemize}
		\end{proof}
		\end{adjustwidth}
		Die Behauptung im Fall $\varphi = \id_K$ ist genau die Aussage von \propref{1_3_13}.
\end{proof}
\begin{remark}
	\proplbl{1_3_14}
	Ist $M\mid K$ eine Erweiterung, die einem Zerfällungskörper $L$ von $f$ enthält, dann ist dieser nicht nur bis auf die Isomorphie sondern als Teilkörper eindeutig bestimmt: $L = K(\alpha_1, \dots, \alpha_n)$, wobei $\alpha_1, \dots, \alpha_n$ genau die $n$ Nullstellen von $f$ in $M$ sind.
\end{remark}
\include{./TeX_files/Korper/Algebraischer_Abschluss}
\include{./TeX_files/Korper/transzendente_Erweiterung}
\section{Separable Polynome}
Sei $K$ ein Körper, $f \in K[X]$, $n = \deg(f)$.
\begin{definition}
	Sei $a \in K$.
	\begin{enumerate}[label={(\arabic*)}]
		\item $\mu(f,a) := v_{x-a}(f) := \sup \set{k \in \N_0 : (x-a)^k \mid f} \in \N_0 \cup \set{\infty}$ die \begriff{Vielfachheit} der Nullstelle $a$ von $f$
		\item Nullstelle $a$ von $f$ ist \begriff{einfach} :$\Leftrightarrow$ $\mu(f,a) = 1$
		\item $f$ ist \begriff{separabel} :$\Leftrightarrow$ jede Nullstelle $a\in\bar K$ von $f\in\bar K[X]$ ist einfach.
	\end{enumerate}
\end{definition}
\begin{remark}
	\begin{enumerate}[label={(\alph*)}]
		\item Ist $L\mid K$ eine Körpererweiterung und $g\in K[X]$, so gilt \begin{flalign*}
			\qquad & f\mid g\;\text{in}\; K[X]\quad\Leftrightarrow\quad f\mid g\;\text{in}\; L[X]
		\end{flalign*}
		Insbesondere ist die Nullstelle $\mu_K(f,a) = \mu_L(f,a)$. Wir können deshalb von der Vielfachheit der Nullstelle von $f$ sprechen.
		\item \proplbl{1_6_2_b} $\displaystyle\#\{a\in K\mid f(a) = 0\} \le \sum_{a\in K} \mu(f,a) \le \sum_{a\in \bar K} \mu(f,a) = \deg(f)$, falls ($f\neq 0$)
		\item Aus \ref{1_6_2_b} folgt insbesondere:\\
		\begin{tabularx}{\linewidth}{XcX}
			\hfill$f$ ist separabel & $\Leftrightarrow$ & $f$ hat genau $\deg(f)$ paarweise verschiedene Nullstellen in $\bar K$
		\end{tabularx}
	\end{enumerate}
\end{remark}
\begin{definition}
	Die \begriff{formale Ableitung} von $f = \sum_{i=1}^n a_i X^{i-1}$ ist \begin{flalign*}
		\qquad &f' := \frac{\d}{\d x} f(x) := \sum_{i=1} i a_i X^{i-1} &
	\end{flalign*}
\end{definition}
\begin{lemma}
	Für $f,g \in K[X], a,b \in K$ gelten\begin{enumerate}[label={(\alph*)}]
		\item $(af + bg)' = a f' + b g'$ (Linearität)
		\item $(fg)' = f'g + fg'$ (Produktregel)
		\item $(f(g(x)))' = f'(g(x))\cdot g'(x)$ (Kettenregel)
	\end{enumerate}
\end{lemma}
\begin{proof}
	Übung.
\end{proof}
\begin{lemma}
	\proplbl{1_6_5}
	Sei $f \neq 0$. Für $a \in K$ gilt
	\begin{flalign*}
		\qquad&\mu(f' , a) \ge \mu(f,a) - 1&
	\end{flalign*}
	mit Gleichheit genau dann, wenn $\chara(K) \nmid \mu(f,a)$.
\end{lemma}
\begin{proof}
	Schreibe $f = (X-a)^k \cdot g$, $k = \mu(f,a)$
	\begin{itemize}[topsep=-.5em,left=3.5em]
		\item[$k=0$:] $\mu(f', a) \ge 0 > -1$ und $\chara(K) \mid 0$
		\item[$k>0$:] $f' = k(X-a)^{k-1}g + (X-a)^k \cdot g'$ $\implies \mu(f',a) \ge k$, sowie 
		
		\vspace*{-2pt}
		\begin{tabular}{@{}>{$}r<{$}>{$}c<{$}>{$}l<{$}}
			\mu(f', a) \ge k & \Leftrightarrow & (X-a)^k \mid k(X-a)^{k-1}\cdot g \\
							 & \Leftrightarrow & X-a \mid k\cdot g \\
							 & \Leftrightarrow & X-a\mid k \\
							 & \Leftrightarrow & k=0\;\text{in}\; K\\
							 & \Leftrightarrow & \chara(K) \mid k
		\end{tabular}
	\end{itemize}
\end{proof}

\begin{proposition}
	\proplbl{1_6_6}
	Sei $f\neq 0$. Dann gilt:
	
	\begin{tabular}{@{$\qquad$}rcl}
		$f$ separabel & $\Leftrightarrow$ & $\ggT(f,f') = 1$
	\end{tabular}
\end{proposition}

\begin{proof}\leavevmode\vspace*{\dimexpr-\baselineskip+2\lineskip}
	\begin{itemize}
		\item[($\Rightarrow$)] $f$ separabel \\[-0.2em]
			\begin{tabular}[t]{@{}>{$}r<{$}@{$\;$}l}
			\Rightarrow & $f = c\cdot\prod_{i=1}^{n}(X-a_i)$ mit $c\in K$, $a_1$, $\dots$, $a_n\in \bar K$ paarweise verschieden und $\mu(f,a_i) = 1$ \\
			\xRightarrow[\chara(K)\nmid 1]{\propref{1_6_5}} & $\mu(f',a_i) = 0$ $\forall i$\\
			\Rightarrow & $\displaystyle\ggT(f,f') = \prod_{a\in\bar K} (X-a)^{\min\{ \mu(f,a), \mu(f',a) \}} = 1$
		\end{tabular}
		\item[($\Leftarrow$)] $f$ nicht separabel $\Rightarrow$ $\exists a\in\bar K$ mit $\mu(f,a)\ge 2$ $\xRightarrow{\propref{1_6_5}}$ $\mu(f',a)\ge 1$.
		
		Mit $g = \MinPol(a\mid K)$ gilt: $g\mid f$ $\Rightarrow$ $\ggT(f,f') \neq 1$
	\end{itemize}
\end{proof}

\begin{lemma}
	$f' = 0$ $\Leftrightarrow$ $\exists g\in K[X]$ mit $f(X) = g(X^p)$ und $p=\chara(K)$.
\end{lemma}
\begin{proof}
	Ist $f = \sum_{i=1}^{n} a_iX^i$ $\Rightarrow$ $f' = \sum_{i=1}^{n}i a_{i-1}X^{i-1}$ und \\[-0.3em]
	\begin{tabular}{@{}r>{$}c<{$}l}
		$f' = 0$	& \Leftrightarrow & $i a_i = 0$ in $K$ $\forall i$\\
					& \Leftrightarrow & $\forall i$: $i = 0$ in $K$ oder $a_i = 0$ \\
					& \Leftrightarrow & $f = a_0 + a_p X^p + \dots + a_{pm} X^{pm} = g(X^p)$ mit $g = a_0 + a_p X + \dots + a_{pm} X^m$
	\end{tabular}
\end{proof}

%TODO
\begin{conclusion}
	\proplbl{1_6_8}
	Sei $f$ irreduzibel
	\begin{enumerate}[label={(\alph*)}]
		\item Ist $\chara(K) = 0$, so ist $f$ separabel
		\item Ist $\chara(K) = p>0$, so sind äquivalent
		\begin{enumerate}[label={(\arabic*)}]
			\item $f$ ist inseparabel
			\item $f' = 0$
			\item $f(X) = g(X^p)$ für ein $g \in K[X]$
		\end{enumerate}
	\end{enumerate}
\end{conclusion}
\begin{proof}
	$f$ irreduzibel $\implies \underbrace{\ggT(f,f') \sim 1}_{\xLeftrightarrow{\propref{1_6_6}} f \text{ sep}} \oder \underbrace{\ggT(f,f') \sim f}_{\xLeftrightarrow{\propref{1_6_6}} f \text{ sep.}}$.
	
	Da $\deg(f') = \deg(f)$ ist
	\begin{flalign*}
		\qquad & f \mid f' \quad \iff \quad  f' = 0 \quad \iff \quad f(X) = g(X^p) \; \text{für ein}\;g &
	\end{flalign*}
	Im Fall $\chara(K) = 0$ tritt dieser Fall nicht ein.
\end{proof}
\begin{definition}[vollkommen]
	$K$ ist \begriff{vollkommen} $\iff$ jedes irreduzibel $f \in K[X]$ ist separabel.
\end{definition}
\begin{example}
	\begin{expenum}
		\item \proplbl{1_6_12_a} $\chara(K) = 0 \implies K$ ist vollkommen
		\item $K = \bar{K} \implies K$ ist vollkommen
		\item $K = \F_p (t)$ ist nicht vollkommen:
		\begin{flalign*}
			\qquad f &= X^p - t \in K[X] \text{ ist irreduzibel} &\\
			f' &= pX^{p-1} = 0 \implies f \text{ nicht seperabel.} &
		\end{flalign*}
		Tatsächlich hat $f$ nur eine Nullstelle in $\bar{K}$: $f = X^p - t \overset{\text{V1}}{=} (X - t^{\frac{1}{p}})^p.$
	\end{expenum}
\end{example}
\begin{definition}
	Sei $\chara(K) = p > 0$.
	\begin{enumerate}[label={(\arabic*)}]
		\item Der \person{Frobenius}-Endomorphismus von $K$ ist
		\begin{flalign*}
		\qquad &\Phi_p\colon \left\lbrace\begin{array}{@{}l@{\;}c@{\;}l}
		K &\to& K\\
		X &\mapsto& X^p 
		\end{array}\right. &
		\end{flalign*}
		\item $K^p = \Image(\Phi_p) = \set{a^p \mid a \in K}$
	\end{enumerate}
\end{definition}
\begin{proposition}
	Sei $\chara(K) = p > 0$. Dann ist $\Phi_p \in \End(K): =\Hom(K,K)$
\end{proposition}
\begin{proof}
	Für $a, b \in K$ ist
	\begin{itemize}[topsep=-6pt]
		\item $\Phi_p = (ab)^p = a^p \cdot b^p = \Phi_p (a) \cdot \Phi_p(b)$
		\item $\Phi_p(a+b) = (a+b)^p = \sum_{i=0}^p\binom{p}{i} a^i b^{p-i} = b^p + a^p = \Phi_p(a) + \Phi_p(b)$, da $p \mid \binom{p}{i}$ für $i = 1, \dots, p-1$ (V1).
		\item $\Phi_p(1) = 1^p = 1$
	\end{itemize}
\end{proof}
\begin{remark}
	\begin{remarkenum}
		\item \proplbl{1_6_13_a} Da $\Phi_p \in \End(K) $ ist $K^p$ ein Teilkörper von $K$ und $\Phi_p$ ist injektiv.
		\item Insbesondere gibt es zu jedem $a \in K$ ein eindeutig bestimmtes $a^{\frac{1}{p}} \in \bar{K}$ mit
		\begin{flalign*}
			\qquad & \Phi_p(a^{\frac{1}{p}}) = (a^{\frac{1}{p}})^p = a &
		\end{flalign*}
		\item Für $a \in \F \cong \F_p$ ist $\Phi_p(a) = a$. (z.B. $\Phi_p(1) = 1$ oder kleiner Satz von \person{Fermat})
	\end{remarkenum}
\end{remark}
\begin{lemma}
	\proplbl{1_6_14}
	Sei $\chara(K) = p > 0$, $a \in K \setminus K^p$. Dann ist $f = X^p -a$ irreduzibel und inseparabel
\end{lemma}
\begin{proof}
	Sei $\alpha \in \bar{K}$ mit $f(\alpha) = 0$, $g= \MinPol(\alpha \mid K)$
	\begin{itemize}[topsep=-6pt]
	\item[$\implies$] $g \mid f = X^p - \alpha = (X-\alpha)^p$
	\item[$\implies$] $g \equiv (X - \alpha)^k$ mit $k \le p$. 
	\end{itemize}
	\medskip
	$a \notin K^p$
	\begin{itemize}[topsep=-6pt,widest=$\xRightarrow{g \text{ irred.}}$,leftmargin=*]
	\item [$\implies$] $\alpha \notin K \implies k >1$
	\item[$\implies$] $g$ ist inseperabel
	\item[$\xRightarrow{g \text{ irred.}}$] $g(X) = h(X^p)$ für ein $h$
	\item[$\implies$] $k = p$ $\implies f = g$ irreduzibel 
	\end{itemize}
\end{proof}

\begin{proposition}
	\proplbl{1_6_15}
	Genau dann ist $K$ vollkommen, wenn \begin{enumerate}[label={(\roman*)}]
		\item $\chara(K) = 0$ oder
		\item $\chara(K) = ß > 0$ und $K^P = K$
	\end{enumerate}
\end{proposition}
\begin{proof}
\leavevmode
\begin{itemize}[topsep=-6pt]
\item $\chara(K) = 0$: klar (\cref{1_6_12_a})
\item $\chara(K) = p > 0$: \begin{itemize}
	\item[($\Rightarrow$)] Es existiert ein $a\in K\setminus K^p$, so ist $K$ nicht vollkommen nach \cref{1_6_14}.
	\item[($\Leftarrow$)] Sei $f(X)\in K[X]$ irreduzibel und inseparabel. Nach \cref{1_6_8} existiert ein $g(X)\in K[X]$ mit \begin{flalign*}
		\qquad & f(X) = g(X^p) &
	\end{flalign*}
	Setze $g(X) = \sum_{i=0}^n a_i X^i\in K[X]$. Dann ist \begin{flalign*}
		\qquad & f(X) = g(X^p) = \sum_{i=0}^n a_i \big(X^i)^p \overset{=}{\text{V1}} \Bigg(\sum_{i=0}^n \underbrace{a_i^{1\!\slash\!n}}_{\mathclap{\text{$\in K$ da $K^p=K$}}} X^i\Bigg)^p, &
	\end{flalign*}
	folglich ein Widerspruch.
	\end{itemize}
\end{itemize}
\end{proof}
\begin{example}
	\proplbl{1_6_16}
	$K$ endlich $\Rightarrow$ $K$ vollkommen (\propref{1_6_13_a}, \propref{1_6_15}).
\end{example}
\include{./TeX_files/Korper/separable_Erweiterungen}
\section{Norm und Spur}
Sei $L\mid K$ endliche Körpererweiterung und $\alpha \in L$.
\begin{remark}
	\proplbl{1_8_1}
	$L$ ist ein $K$-Vektorraum $\implies$ $\End_K (L)$ ist ein $K$-Vektorraum und ein (nicht kommutativer) Ring unter Komposition.
\end{remark}
\begin{definition}[Spur, Norm]
	\begin{enumerate}[label={(\alph*)}]
		\item 
		\bgroup
		\zeroAmsmathAlignVSpaces*[-0.5\baselineskip]
		\begin{flalign*}
		 \;\;&\mu_{\alpha}\colon \begin{cases}
		L & \to L\\
		x &\mapsto \alpha x
		\end{cases} \in \End_K (L) &
		\end{flalign*}
		\egroup
		\item \begin{enumerate}[label=,left=0pt]
			\item $N_{L \mid K}(\alpha) := \det(\mu_{\alpha}$, die $(L \mid K)$- Norm von $\alpha$
			\item $\Tr_{L \mid K}(\alpha) := \Tr(\mu_{\alpha})$, die $(L\mid K)$-Spur von $\alpha$
		\end{enumerate}
		\item \begin{enumerate}[label=,left=0pt]
			\item $\chi_{\alpha} :=$ charakteristisches Polynom von $\mu_{\alpha}$
			\item $f_{\alpha} :=$ Minimalpolynom von $\mu_{\alpha}$
		\end{enumerate}
	\end{enumerate}
\end{definition}
\begin{lemma}
	\proplbl{1_8_3}
	\begin{enumerate}[label={(\alph*)}]
		\item $f_{\alpha} = \MinPol(\alpha \mid K)$
		\item $\chi_{\alpha} = f_{\alpha}^m$ für $m = [L:K(\alpha)]$
	\end{enumerate}
\end{lemma}
\begin{proof}\leavevmode
	\begin{enumerate}[topsep=-6pt,left=0pt,label={(\alph*)}]
		\item Die Abbildung
		\begin{flalign*}
			\quad&\mu\colon \left\lbrace\begin{array}{@{}l@{\;}c@{\;}l}
			L & \to &\End_K(L)\\
			\beta & \mapsto& \mu_{\beta} 
			\end{array}\right. & \tag{$\star$} \proplbl{bew:1_8_3}
		\end{flalign*}
		ist $K$-linearer Ringhomomorphismus: \checkmark\\
		Sei $g:= \MinPol(\alpha \mid K)$. Dann
		\begin{flalign*}
			\quad &\left.\begin{array}{@{}l@{\,}c@{\,}l@{\;}l@{\;}l@{\;\;}c@{\;\;}l@{\;}l}
			g(\mu_{\alpha}) & \overset{\eqref{bew:1_8_3}}{=} & \mu_{g(\alpha)} & = & 0 \in \End_K(L) &&& \implies  f_{\alpha}\mid g\\
			\mu_{f_{\alpha}(\alpha)} & \overset{\eqref{bew:1_8_3}}{=} & f_{\alpha}(\mu_{\alpha}) & = & 0 \in \End_K(L) & \xRightarrow{\mu \text{ inj.}} & f_{\alpha}(\alpha) = 0 & \implies g \mid f_{\alpha}
			\end{array}\right\rbrace \implies f_{\alpha} = g &
		\end{flalign*}
		\item Charakteristisches Polynom und Minimalpolynom haben die gleichen irreduziblen Faktoren: $\nearrow$ LAAG VIII.7.6 oder direkt:
		\vspace*{\dimexpr-\baselineskip-2\lineskip}
		\begin{adjustwidth}{1.5em}{0pt}
			\item $V$ $n$-dimensionaler $K$-VR, $\varphi \in \End_K(V)$, $\mathscr{B}$ Basis von $V$ $\rightsquigarrow$ $A = M_{\mathscr{B}}(f)$. $\chi_{\varphi} = \chi_A \in K[X]$ zerfällt in Linearfaktoren in $\bar{K}[X]$\\
			$\implies$ lese $\chi_{\varphi} = \chi_A \und P_{\varphi} = P_A$ aus der Jordan-Normalform von $A$ ab.
		\end{adjustwidth}
		$f_{\alpha} = \MinPol(\alpha\mid K)$ irreduzibel $\implies \chi_{\alpha} = f_{\alpha}^m$ für ein $m$ und
		\begin{flalign*}
			\quad&\left .\begin{array}{@{}l@{\,}c@{\,}l@{\,}c@{\,}l}
			\deg(f_{\alpha}) &=& \deg(\alpha \mid K) &=& [K(\alpha) : K]\\
			\deg(\chi_{\alpha}) &=& \mathrm{dim}_{\mathrm K}\, L &=& [L:K]	
			\end{array}\right\rbrace \implies m = \frac{\deg(\chi_{\alpha})}{\deg(f_{\alpha})} = \frac{[L:K]}{[K(\alpha):K]} = [L:K(\alpha)] &
		\end{flalign*}
	\end{enumerate}
\end{proof}
\begin{example}
	\proplbl{1_8_4}
	Sei $\C = \R + \R\ii$, $\alpha = x+y\ii \in \C$.
	\begin{itemize}[topsep=-6pt]
		\item[$\Rightarrow$] $\mu_\alpha$ bezüglich Basis $(1,i) = \mathcal B$ ist \begin{flalign*}
			\quad & M_{\mathcal B}(\mu_\alpha) = \begin{pmatrix}
				x & -y \\ y & x
			\end{pmatrix} &
		\end{flalign*}
		\item[$\Rightarrow$] $\begin{aligned}[t]N_{\mathbb C\mid\mathbb R} (\alpha)\; &= \det\big(M_{\mathcal B}(\mu_\alpha)\big) = x^2 + y^2 = \vert \alpha\vert^2 = \alpha\bar\alpha,\\
		\mathrm{Sp}_{\mathbb C\mid\mathbb R}(\alpha) &= \mathrm{Sp}\big( M_{\mathcal B}(\mu_\alpha)\big) = 2\alpha = \alpha + \bar\alpha\end{aligned}$\\[4\lineskip]
		$\begin{aligned}[t]\chi_\alpha(t) &= \det(\mathbbm 1 - A) = (t - x)^2 + y^2 = t^2- 2xt + x^2 + y^2 = t^2 - 2 \mathrm{Sp}_{\mathbb C\mid\mathbb R}(\alpha)t + N_{\mathbb C\mid\mathbb R}(\alpha)\\
		& = (t - \alpha)(x - \bar\alpha),\\
		\displaystyle f_\alpha(t) &= \begin{cases}
			t - \alpha, & \alpha\in\mathbb R,\\ (t-\alpha)(t-\bar\alpha), & \alpha\notin\mathbb R
		\end{cases}\end{aligned}$
	\end{itemize}
\end{example}

\begin{lemma}
	\proplbl{1_8_5}
	Seien $n = [L:K]$ und $\alpha$,$\beta\in L$, $\lambda\in K$.
	\begin{enumerate}[label={(\alph*)}]
		\item $N_{L\mid K}(\alpha\beta) = N_{L\mid K}(\alpha)\cdot N_{L\mid K}(\beta)$,
		\item $\mathrm{Sp}_{L\mid K}(\lambda\alpha + \beta) = \lambda\,\mathrm{Sp}_{L\mid K}(\alpha) + \mathrm{Sp}_{L\mid K}(\beta)$,
		\item $N_{L\mid K}(\lambda) = \lambda^n$, $\mathrm{Sp}_{L\mid K}(\lambda) = n\cdot\lambda$,
		\item Ist $f_\alpha = X^r + a_{r-1} X^{r-1} + \dots + a_0$ und $m = [L:K(\alpha)] = n\mskip-1mu\big\slash \mskip-1mu r$, so ist \begin{flalign*}
			\quad & N_{L\mid K}(\alpha) = (-1)^n a_0^m,\quad\mathrm{Sp}_{L\mid K}(\alpha) = -m a_{r-1} &
		\end{flalign*}
	\end{enumerate}
\end{lemma}
\begin{proof}\leavevmode
	\begin{itemize}[topsep=-6pt,widest={(a), (b)},leftmargin=*]
		\item[(a), (b)] klar: Multiplikativität der Determinante und Linearität der Spur
		\item[(c)] $M_{\mathcal B}(\mu_\lambda) = \lambda\mathbbm 1$ für alle Basen $\mathcal B$ von $L$.
		\item[(d)] $\chi_\alpha = X^n + b_{n-1} X^{n-1} + \dots + b_0$ \begin{itemize}[topsep=-6pt]
			\item[$\Rightarrow$] $\det\mu_\alpha = (-1)^n \chi_\alpha(0) = (-1)^n b_0$, $\mathrm{Sp}_{\mu_\alpha} = - b_{n-1}$
		\end{itemize}
		\vspace*{4\lineskip}
		$\chi_\alpha = \big(f_{\alpha}\big)^m$
		\begin{itemize}[topsep=-6pt]
			\item[$\Rightarrow$] $N_{L\mid K}(\alpha) = \det(\mu_\alpha) = (-1)^n a_0^m$, $\mathrm{Sp}_{L\mid K}(\alpha) = \mathrm{Sp}\,\mu_\alpha = -b_{n-1} = -m \cdot a_{n-1}$
		\end{itemize}
	\end{itemize}
\end{proof}

\begin{remark}
	\proplbl{1_8_6}
	\begin{enumerate}[left=0pt,label={(\alph*)}]
		\item Ist $\alpha$ inseparabel über $K$, so ist $f_\alpha(X) = g(X^r)$ für ein $g\in K[X]$, und somit ist \begin{flalign*}
			\quad & \mathrm{Sp}_{L\mid K}(\alpha) = -m\cdot \underbrace{a_{r-1}}_{=0} = 0 &
		\end{flalign*}
		\item Ist $L\mid K(\alpha)$ inseparabel, so ist $m = p^d\cdot [L:K(\alpha)]_{\mathrm S}$, somit ist ist \begin{flalign*}
			\quad & \mathrm{Sp}_{L\mid K}(\alpha) = \underbrace{m}_{\mathclap{=0}}\cdot\, a_{r-1} = 0 &
		\end{flalign*}
		\item Aus $(a)$ und $(b)$ folgt: \begin{adjustwidth}{2em}{0pt}
			$L\mid K$ inseparabel \hspace*{1em} $\Leftrightarrow$ \hspace*{1em} $\mathrm{Sp}_{L\mid K} = 0$
		\end{adjustwidth}
	\end{enumerate}
\end{remark}

\begin{proposition}
	\proplbl{1_8_7}
	Ist $\alpha\in L$, $n=[L:K] = q\cdot r$ und $r=[L:K]_{\mathrm S}$ sowie $\hom_K(L,\bar K) = \{ \sigma_1,\dots,\sigma_r \}$, so gilt \begin{flalign*}
		\quad & N_{L\mid K}(\alpha) = \Bigg( \prod_{r=1}^n \sigma_j(\alpha)\Bigg)^q,\qquad \mathrm{Sp}\,(\alpha) = q\sum_{i=1}^{r} \sigma_i(\alpha) &
	\end{flalign*}
\end{proposition}
\begin{proof}
	Sei $n_1 = [K(\alpha): K] = r_1 q_1$ und $n_2 = [L:K(\alpha)] = r_2 q_2$. Schreibe \begin{flalign*}
		\quad & f_\alpha = X^{r_1} + a_{n_1 - 1} X^{n_1 - 1} + \dots + a_0 = \prod_{i=1}^{r_1} \big( X - \tau_i(\alpha)\big)^{q_1} = g\big( X^{q_1}\big),\quad g(X) = \prod_{i=1}^{r_1} \big( X - \tau_i^{q_1}(\alpha)\big) &
	\end{flalign*}
	Jedes $\tau_i$ hat genau $r_2$ viele Fortsetzungen zu einem $\sigma_j \in \hom_K(L\mid \bar K)$ (\propref{1_7_3}), sodass \begin{flalign*}
		\quad & \Bigg(\prod_{j=1}^r \sigma_j(\alpha)\Bigg)^q = \Bigg( \prod_{i=1}^{r_1} \tau_i(\alpha)^{r_2}\Bigg)^q = \big( (-1)^{r_1} a_0\big)^{r_2 q_2} = (-1)^n a_0^{r_2} \overset{\propref{1_8_5}}{=} N_{L\mid K}(\alpha), \\
		& q\sum_{i=1}^r \sigma_j(\alpha) = q r_2 \sum_{i=1}^{r_1} \tau_j(\alpha) = - q_2 r_1 a_{n_1 - 1} = \mathrm{Sp}_{L\mid K}(\alpha) &
	\end{flalign*}
\end{proof}

%to add 16 May 2019 %TODO
\begin{lemma}
	\proplbl{1_8_8}
	Seien $K \subseteq L \subseteq M$ Körper mit $M\mid K$ endlich und sei $\alpha\in M$. Dann ist
	\begin{itemize}
		\item $N_{M \mid K}(\alpha) = N_{L\mid K}\big(N_{M\mid L}(\alpha)\big)$
		\item $\Tr_{M \mid K}(\alpha) = \Tr_{L \mid K}\big(\Tr_{M \mid L}(\alpha)\big)$
	\end{itemize}
\end{lemma}
\begin{proof}
	Sei $[L:K] = q_1 \cdot r_1$, $[M:L] = q_2 \cdot r_2$; $\Hom(M, \bar{L}) = \set{\sigma_1, \dots, \sigma_{r_2}}$. Fixiere die Einbettung $L \subseteq \bar{K}$ und setze $\tau_i$ fort zu $\tilde{\tau}_i \in \Aut(\bar{K}\mid K)$( \propref{1_4_11}) %TODO
	Dann ist
	\begin{flalign*}
		\quad & \Hom_K(M, \bar{K}) = \big\lbrace \tilde{\tau}_i \circ \sigma_j\;\big|\; i =1, \dots, r_1,\; j = 1, \dots, r_2\big\rbrace, &
	\end{flalign*}
	denn $\# \Hom(M, \bar{K}) = [M: K]_{\mathrm S} = r_1 \cdot r_2$ und
	\begin{flalign*} %TODO check for { } misssing
		\quad & \begin{aligned}[t]
			& \tilde \tau_i\circ \sigma_j = \tilde \tau_{i'}\circ\sigma_{j'} \\
			\Rightarrow \;& \sigma_j = \big( \tilde \tau_i^{-1}\circ \tilde\tau_{i'}\big)\circ \sigma_{j'} \\
			\Rightarrow\; & \tilde \tau_i^{-1} \circ \left. \tilde \tau_i \right| _{L} = \id_L \\
			\Rightarrow\; & \tau_i = \tau_{k'} \quad \Rightarrow i=i' \quad \Rightarrow \sigma_j = \sigma_{j'} \quad\Rightarrow j = j' \\
			\Rightarrow\; & N_{L\mid K}\big( N_{M\mid L}(\alpha)\big) \overset{\propref{1_8_7}}{=} N_{L\mid K}\Bigg( \prod_{j=1}^{r_2} \sigma_i(\alpha) \Bigg)^{q_2} \overset{\propref{1_8_7}}{=} \prod_{i=1}^{r_1} \tilde \tau_i \Bigg( \prod_{j=1}^{r_2} \sigma_j(\alpha)\Bigg)^{q_1 q_2} = \Bigg( \prod_{i,j} \big(\tilde \tau_i \circ\sigma_j\big)(\alpha)\Bigg)^{q_1 q_2} \overset{\propref{1_8_7}}{=} N_{M\mid K}(\alpha)
		\end{aligned} &
	\end{flalign*}
	Analog für die Spur.
\end{proof}
\begin{theorem}[Unabhängigkeit der Charaktere, \person{Artin}]
	\proplbl{1_8_9}
	Sei $G$ eine Gruppe. Sind $\chi_1, \dots, \chi_n \in \Hom(G, K^{\times})$ paarweise verschieden, so sind sie linear unabhängig im $K$-Vektorraum $\Abb(G,K)$.
\end{theorem}
\begin{proof} %TODO reformat!
	Seien $\chi_1, \dots, \chi_n$ linear abhängig, oE $n \ge 2$ minimal, d.h.
	\begin{flalign*}
		\quad & \sum_{i=1}^n a_i \chi_i = 0 \quad\text{mit}\; a_1, \dots, a_n \in K^{\times}. &
	\end{flalign*}
	Sind $\chi_1 \neq \chi_n$ $\implies$ $\exists g \in G$ mit $\chi_1(g) \neq \chi_n(g)$. Ist die Summe $\sum a_i \chi_i = 0$, so  folgt, dass $\forall h \in G$ ist $\sum_{i=1}^n a_i \chi_i (h) = 0$ und
	\begin{flalign*}
		\quad & \begin{aligned}[t]
		\implies & \forall h\in G\colon\;\left\lbrace\begin{array}{@{}l@{\,}c@{\,}l}
			\sum_{i=1}^n a_i \cdot \underbrace{\chi_i (hg)}_{\chi_i(h)\cdot \chi_i(g)} &=& 0\\
			\sum_{i=1}^n a_i \cdot \chi_i(h)\cdot \chi_i(g) &=& 0
		\end{array}\right.\\
		\implies & 0 = \sum_{i=1}^n a_i \cdot \chi_i(h)\big(\chi_i(g) - \chi_n(g)\big) = \sum_{i=1}^{n-1} a_i \big(\chi_i(g) - \chi_n(g)\big)\cdot \chi_i(h)\\
		\implies & \sum_{i=1}^{n-1} a_i \cdot \big(\chi_i(g) - \chi_n(g)\big)\cdot \chi_i = 0
	\end{aligned} &
	\end{flalign*}
	$
	a_n (\chi_1(g) - \chi_n(g)) \neq 0$, was ist ein Widerspruch zur Minimalität von $n$.
\end{proof}
\begin{conclusion}
	\proplbl{1_8_10}
	Genau dann ist $\Tr_{L \mid K} \neq 0$, wenn $L \mid K$ separabel.
\end{conclusion}
\begin{proof}\leavevmode
	\begin{itemize}[topsep=-6pt]
		\item[($\Rightarrow$)] \propref{1_8_6}
		\item[($\Leftarrow$)] Sei $\Hom_K(L, \bar{K}) = \set{\sigma_1, \dots, \sigma_n}$. $\left.\sigma_i\right|_{L^{\times}} \in \Hom_K(L^{\times}, K^{\times})$\\
		$\xRightarrow{\propref{1_8_7}}$ $\sigma_1,\dots, \sigma_n$ sind $\bar{K}$-linear unabhängig. Insbesondere ist $\Tr_{L \mid K} = \sum_{i=1}^n \sigma_i \neq 0$.
	\end{itemize}
\end{proof}
\section{Einfache Erweiterung}
Sei $K$ unendlich, $L \mid K$ endliche Erweiterung.
\begin{remark}
	$L \mid K$ einfach $\Longleftrightarrow L = K(\alpha)$ für ein $\alpha \in L$. Ein solches $\alpha$ heißt ein \begriff{primitives Element} von $L \mid K$.
\end{remark}
\begin{proposition}
	\proplbl{1_9_2}
	\begin{tabularx}{\linewidth}{l@{\quad}c@{\quad}X}
		$L\mid K$ einfach & $\Leftrightarrow$ & Die Menge der Zwischenkörper von $\mathcal M = \lbrace M \mid K \subseteq M \subseteq L\rbrace$ ist endlich.
	\end{tabularx}
	\vspace*{-\baselineskip}
\end{proposition}
\begin{proof}\leavevmode
	\begin{itemize}[topsep=-6pt]
		\item[($\Rightarrow$)] Sei $L = K(\alpha)$, $f= \MinPol(\alpha \mid K)$. Für $M \in \mathcal M$ setze
		\begin{flalign*}
			\quad & \begin{aligned}[t]
				g &:= \MinPol(\alpha\mid M) = \sum_{i=0}^n a_i X^i,\\
				M_0 &:= K(a_0,\dots,a_n).
			\end{aligned}&
		\end{flalign*} 
		Dann gilt $g \mid f$ in $L[X]$, es gibt also nur endlich viele solche $g$. Da $K \subseteq M_0 \subseteq M \subseteq L$ und
		\begin{flalign*}
			\quad & [L:M_0] = [M_(\alpha):M_0]= \deg(g) = [M(\alpha):M] = [L:M] &
		\end{flalign*}
		ist $M = M_0$ durch $g$ bestimmt.
		\item[($\Leftarrow$)] Sei $L = K(\alpha_1, \dots, \alpha_r)$. Es genügt, die Behauptung für $r = 2$ zu zeigen. Sei also $L = K(\alpha, \beta)$, oE $\beta \neq 0$. Da $\abs{K} = \infty$ ist $\abs{\set{\alpha + c\beta \mid c \in K}} = \infty$. Ist $\abs{\mathcal M} < \infty$, so existiert somit $c$, $c' \in K$ mit $c \neq c'$ und $K(\alpha + c \beta) = K(\alpha + c' \beta) =: M \in \mathcal M$
		\begin{itemize}[topsep=-6pt]
		\item [$\implies$] $M \ni (\alpha + c \beta) \cdot (\alpha + c' \beta) = (\underbrace{c-c'}_{\in K^{\times}})\beta$
		\item [$\implies$] $\beta \in M \implies \alpha \in M$
		\item [$\implies$] $L = K(\alpha, \beta) \subseteq M \subseteq L$
		\item [$\implies$] $L = M = K(\alpha + c\beta)$.
		\end{itemize}
	\end{itemize}
\end{proof}
\begin{remark}
	\begin{enumerate}[label={(\alph*)}]
		\item Insbesondere gilt:
		$K \subseteq M \subseteq L$, $L \mid K$ endlich und einfach\\
		$\implies M \mid K$ endlich und einfach
		\item Dies gilt auch für transzendente einfache Erweiterungen. $K \subseteq M \subseteq L = K(X) \implies M = K(f)$ für ein $f \in K(X)$. ($\nearrow$ Satz von \person{Lüroth})
	\end{enumerate}
\end{remark}
\begin{theorem}[Satz vom primitiven Element, \person{Abel}]
	\proplbl{1_9_4}
	Sei $L = K(\alpha_1, \dots, \alpha_r)$ eine endliche Erweiterung von $K$. Ist höchstens eines der $\alpha_i$ inseparabel über $K$, so ist die $L \mid K$ einfach.
\end{theorem}
\begin{proof}
	Es genügt, den Fall $r = 2$ zu betrachten (\propref{1_7_6}). Sei also $L = K(\alpha,\beta)$ und $\beta$ sei separabel über $K$. Seien \begin{equation*}
		\alpha = \alpha_1,\dots,\alpha_n,\;\beta = \beta_1,\dots,\beta_l
	\end{equation*}
	die zu $\alpha$ bzw. $\beta$ $K$-Konjugierten. Da $\vert K \vert = \infty$ existiert ein $c\in K$ mit \begin{equation*}
		c \neq \frac{\alpha_i - \alpha}{\beta-\beta_j},\quad i=1,\dots,n,\;j=2,\dots,l
	\end{equation*}
	Sei $\gamma := \alpha + c\beta$ und $f = \MinPol(\alpha\mid K)$ sowie $g := \MinPol(\beta\mid K)$.
	\begin{underlinedenvironment}[Behauptung]
		$g(X)$ und $f(\gamma- cX)$ haben genau eine gemeinsame Nullstelle $\beta$.
	\end{underlinedenvironment}
	\vspace*{\dimexpr-\baselineskip-2\lineskip}
	\begin{proof}\leavevmode
		\begin{itemize}[topsep=\dimexpr-6pt-\baselineskip-4\lineskip\relax]
			\item $g(\beta) = 0$, $f(\gamma - c\beta) = f(\alpha) = 0)$
			\item $f(\gamma - c\beta_j) = 0$ \begin{itemize}[topsep=-6pt,label={$\Rightarrow$}]
				\item $\exists\,i\colon$ $\alpha + c(\beta - \beta_j) = \alpha_i$
				\item $\displaystyle c = \frac{\alpha_i - \alpha}{\beta - \beta_j}$
				\item Entweder ein Widerspruch oder $j=1$
			\end{itemize}
		\end{itemize}
	\end{proof}
	Sei $h := \MinPol(\beta\mid K(\gamma))$. Dann gilt $h\mid g$, $h\mid f(\gamma -cX)$ \begin{itemize}[topsep=-6pt,label={$\Rightarrow$},widest={$\xRightarrow{\text{$\beta$ sep.}}$},leftmargin=*]
		\item[$\xRightarrow{\text{Beh.}}$] $h$ hat nur eine Nullstelle in $\bar K$
		\item[$\xRightarrow{\text{$\beta$ sep.}}$] $g$ separabel
		\item $\deg(h) = 1$
		\item $\beta\in K(\gamma)$ $\Rightarrow$ $\alpha\in K(\gamma)$
		\item $L = K(\alpha,\beta) = K(\gamma)$
	\end{itemize}
\end{proof}
\begin{conclusion}
	\proplbl{1_9_5}
	Jede endliche separable Erweiterung von $K$ ist einfach und besitzt nur endliche viele Zwischenkörper. Dies gilt insbesondere für jede endliche Erweiterung in Charakteristik 0.
\end{conclusion}
\begin{proof}
	Folgt aus \propref{1_9_2}, \propref{1_9_4} und \propref{1_6_15}.
\end{proof}
\begin{example}
	$\Q(\sqrt{2}, \sqrt{3})\mid \Q$ besitzt ein primitives Element, z.B. $\sqrt{2} + \sqrt{3}$ ($\nearrow$ Übung 21). Tatsächlich ist $\Q(\sqrt{2}, \sqrt{3}) = \Q(\sqrt{2}+c\sqrt{3})$ für jedes $c \in \Q^{\times}$.
	
	\begin{tabular}{@{}l@{ }l@{: }l}
		$K$-Konjugierte zu & $\sqrt 2$ & $\pm\sqrt 2$\\[-0.7em]
						   & $\sqrt{3}$ & $\pm \sqrt3$
	\end{tabular}

	Folglich ist \begin{equation*}
		\Bigg\lbrace \frac{\alpha_i - \alpha}{\beta-\beta_j} \;\Bigg|\; i = 1,2,\;j = 2\Bigg\rbrace = \bigg\lbrace 0, \frac{-2\sqrt{3}}{2\sqrt{3}}\bigg\rbrace
	\end{equation*}
	die Menge der nicht-zugelassenen Proportionalitätsfaktoren und $\alpha + c\beta$ ist primitives Element für alle $c\in \mathbb Q\setminus \lbrace 0, -\sqrt{2}\mskip-1mu\big\slash\mskip-1mu\sqrt3\rbrace = \mathbb Q^\times$
\end{example}
\begin{example}
	Sei $L = \F_p(t,s) = \Quot(\F_p[t,s])$, $K = L^p$. Dann ist $[L:K] = p^2$ ($\nearrow$ P41) aber $L\mid K$ ist \emph{nicht} einfach und besitzt unendliche viele Zwischenkörper. (Nach \propref{1_9_2}) ($\nearrow$ Übung)
\end{example}
\begin{remark}
	Das \propref{1_9_4} gilt auch für $K$ endlich, siehe II.3. %TODO ref later!
\end{remark}

\chapter{Galoistheorie}
\section{Normale Körpererweiterungen}
Sei $K$ Körper, $\bar K$ ein fixierter algebraischer Abschluss von $K$ und $L$ ein Zwischenkörper $K\subseteq L\subseteq \bar K$.

\begin{definition}
	$L\mid K$ ist \begriff{Normal} :$\Leftrightarrow$ Ist $\alpha\in L$ und $\beta\in\bar K$ $K$-konjugiert, so ist $\beta\in L$.
\end{definition}

\begin{proposition}
	\proplbl{2_1_2}
	Ist $L\mid K$ endlich, so sind äquivalent \begin{propenum}
		\item $L\mid K$ ist normal
		\item Jedes irreduzible $f\in K[X]$, das eine Nullstelle in $L$ hat, zerfällt über $L$ in Linearfaktoren
		\item $L$ ist der Zerfällungskörper von $f\in K[X]$
		\item \proplbl{2_1_2_4} Für jedes $\sigma\in\Aut(\bar K\mid K)$ ist $\sigma(L) = L$
		\item Jedes $\sigma\in\Aut(\bar K\mid K)$ ist $\sigma(L)\subseteq L$
	\end{propenum}
\end{proposition}

\begin{proof}\leavevmode
	\begin{itemize}[widest={(1) $\Rightarrow$ (2)},leftmargin=*,topsep=-6pt]
		\item[(1) $\Rightarrow$ (2)] klar nach \propref{1_4_14}
		\item[(2) $\Rightarrow$ (3)] Sei $L = K(\alpha_1,\dots,\alpha_n)$. Mit \begin{equation*}
			f = \prod_{i=1}^n \MinPol(\alpha_i\mid K)
		\end{equation*}
		ist $L$ der Zerfällungskörper von $f$.
		\item[(3) $\Rightarrow$ (4)] Ist $f$ der Zerfällungskörper von\begin{equation*}
			f = \prop_{i=1}^n (X - X_i),
		\end{equation*}
		und $\sigma\in\Aut(\bar K\mid K)$, so permutiert $\sigma$ die Nullstellen $\lbrace \alpha_1,\dots,\alpha_n\rbrace$ von $f$, folglich \begin{equation*}
			\sigma(L) = \sigma\big( K(\alpha_1,\dots,\alpha_n)\big) = K\big(\sigma(\alpha_1),\dots,\sigma(\alpha_n)\big) = K(\alpha_1,\dots,\alpha_n) = L.
		\end{equation*}
		\item[(4) $\Rightarrow$ (5)] trivial
		\item[(5) $\Rightarrow$ (1)] trivial
	\end{itemize}
\end{proof}

\begin{example}
	\begin{enumerate}[label={\alph*)}]
		\item $K\mid K$ ist normal
		\item $\bar K\mid K$ ist normal
		\item $\bar K_{\mathrm S} \mid K$ ist normal (\propref{1_7_7})
		\item $[L:K] = 2$ $\Rightarrow$ $L\mid K$ ist normal
		
		($\deg(f) = 2$, $f$ hat Nullstelle $\Rightarrow$ $f$ zerfällt in Linearfaktoren)
		\item $L = \mathbb Q(\sqrt[3]2)$, $[L:\mathbb Q] = 3$ $L\mid Q$ ist nicht normal, die zu $\sqrt[3]2$ $\mathbb Q$-konjugierte Elemente $\zeta_3 \sqrt[3]2$ und $\zeta_3^2 \sqrt[3]2$ liegen \emph{nicht} in $L$ (\propref{1_3_11_b})
		\item $Sei \alpha = \sqrt[4]2\in\mathbb R_{\ge 0}$  und $f = \MinPol(\alpha\mid\mathbb Q) = X^4 - 2$. Dann sind die $\mathbb Q$-konjugierten $\pm \sqrt[4]2$ und $i\sqrt[4]2$. Da $i\sqrt[4]2\notin\mathbb R$ ist $\mathbb Q(\alpha)\mid \mathbb Q$ nicht normal und \begin{equation*}
			\underbrace{\mathbb Q(\sqrt[4]2) \;\, \underset{\text{normal}}{\overset{2}{\rule[0.1\baselineskip]{3em}{0.1pt}}} \;\, \mathbb  Q(\sqrt 2) \;\, \underset{\text{normal}}{\overset{2}{\rule[0.1\baselineskip]{3em}{0.1pt}}} \;\, \mathbb Q,}_{\text{nicht normal}}
		\end{equation*}
		also ist Normalität nicht transitiv.
	\end{enumerate}
\end{example}

\begin{conclusion}
	\proplbl{2_1_4}
	Sei $L\mid K$ endlich und seien $K\subseteq L_1$, $L_2\subseteq L$ Zwischenkörper. Dann \begin{enumerate}[label={(\alph*)}]
		\item Sind $L_1\mid K$ und $L_2\mid K$ normal, so auch $L_1\cap L_2\mid K$ und $L_1L_2 \mid K$
		\item Ist $L\mid K$ normal, so auch $L\mid L_1$
	\end{enumerate}
\end{conclusion}

\begin{proof}\leavevmode
\begin{enumerate}[label={\alph*)},topsep=-6pt]
	\item \begin{itemize}[left=0pt]
		\item $L1\cap L_2$: klar aus Definition
		\item $L_1L_2$: Sei $\sigma\in\Aut(\bar K\mid K)$ $\Rightarrow$ $\sigma(L_1L_2)  = \sigma(L_1)\sigma(L_2) = L_1 L_2$
	\end{itemize}
	\item klar, da $\Aut(\bar L_1\mid L_1)\subseteq \Aut(\bar K\mid K)$
\end{enumerate}
\end{proof}

\begin{proposition}
	\proplbl{2_1_5}
	Sei $L\mid K$ endlich. Es ist \begin{equation*}
		\# \Aut(L\mid K) \le [L:K]_{\mathrm S}
	\end{equation*}
	mit Gleichheit, wenn die Erweiterung normal ist.
\end{proposition}

\begin{proof}
	Es ist \begin{equation*}
		\Aut(L\mid K) = \Hom_K(L, L) = \big\lbrace \sigma\in\Hom_K(L,\bar K)\;\big|\; \sigma(L)\subseteq L\big\rbrace \subseteq \Hom_K(L,\bar K),
	\end{equation*}
	sodass $\# \Aut(L\mid K) \le \# \Hom_K(L,\bar K) = [L:K]_{\mathrm S}$.
	
	Es gilt: $\Aut(L\mid K) = \Hom_K(L\mid \bar K)$ \begin{itemize}[topsep=0pt,label={$\Leftrightarrow$},widest={<I.4.11>},leftmargin=*]
		\item $\forall \sigma\in \Hom_K(L\mid \bar K)$: $\sigma(L)\subseteq L$
		\item[$\xLeftrightarrow{\propref{1_4_11}}$] $\forall \sigma\in\Aut(\bar K\mid K)$: $\sigma(L)\subseteq L$
		\item[$\xLeftrightarrow{\propref{2_1_2}}$] $L\mid K$ normal.
	\end{itemize}
\end{proof}

\begin{remark}
	\proplbl{2_1_6}
	Es ist also \begin{equation*}
		\Aut(L\mid K) \overset{\circled{\tiny  1}}{\le} [L:K]_{\mathrm S} \overset{\circled{\tiny2}}{\le} [L:K],
	\end{equation*}
	wobei gilt: \begin{enumerate}[label=\protect\circled{\arabic*}]
		\item ist Gleichheit :$\xLeftrightarrow{\propref{2_1_5}}$ $L\mid K$ normal
		\item ist Gleichheit :$\xLeftrightarrow{\propref{1_7_6}}$ $L\mid K$ separabel
	\end{enumerate}
\end{remark}

\begin{definition}
	$L\mid K$ ist \begriff{galoissch} (oder Galoiserweiterung) $\Leftrightarrow$ $L\mid K$ ist normal und separabel
\end{definition}

\begin{proposition}
	\proplbl{2_1_8}
	Ist $L\mid K$ endlich, so sind äquivalent \begin{enumerate}[label={(\arabic*)}]
		\item $L\mid K$ ist galoissch
		\item Jedes $\alpha\in L$ hat $\deg(\alpha\mid L)$ viele $K$-konjugierte in $L$
		\item $L$ ist Zerfällungskörper eines irreduziblen, separablen Polynoms $f\in K[X]$
		\item $L$ ist Zerfällungskörper eines separablen Polynoms $f\in K[X]$
		\item $\#\Aut(L\mid K) = [L:K]$
	\end{enumerate}
\end{proposition}
\begin{proof}\leavevmode
	\begin{itemize}[topsep=-6pt,widest={(1) $\Leftrightarrow$ (3)},leftmargin=*]
		\item[(1) $\Leftrightarrow$ (5)] \propref{2_1_6}
		\item[(1) $\Leftrightarrow$ (2)] $L\mid K$ separabel $\Leftrightarrow$ jdes $\alpha\in L$ hat $\deg(\alpha\mid K)$ viele $K$-konjugierte in $\bar K$. \\
		$L\mid K$ normal $\Leftrightarrow$ alle $K$-konjugierte von $\alpha\in L$ liegen in $L$.
		\item[(1) $\Rightarrow$ (3)] $L\mid K$ separabel $\xRightarrow{\propref{1_9_4}}$ $L = K(\alpha)$ einfach.\\
		$L\mid K$ normal $\Rightarrow$ $L$ ist Zerfällungskörper von $\MinPol(\alpha\mid K)$
		\item[(3) $\Rightarrow$ (4)] trivial
		\item[(4) $\Rightarrow$ (1)] \propref{2_1_2} und \propref{1_7_6}
	\end{itemize}
\end{proof}

\begin{conclusion}
	\proplbl{2_1_9}
	Sei $L\mid K$ endlich und seien $K\subseteq L_1$, $L_2\subseteq L$ Zwischenkörper. \begin{enumerate}[label={(\alph*)}]
		\item Sind $L_1\mid K$ und $L_2\mid K$ galoissch, so auch $L_1\cap L_2\mid K$ und $L_1L_2 \mid K$
		\item Ist $L\mid K$ galoissch, so auch $L\mid L_1$
	\end{enumerate}
\end{conclusion}

\begin{proof}
	\propref{2_1_4}, \propref{1_7_8} und \propref{1_7_9}.
\end{proof}

\begin{definition}
	\proplbl{2_1_10}
	Ist $L\mid K$ galoissch, so heißt \begin{align*}
		\Aut(L\mid K) = \Gal(L\mid K)
	\end{align*}
	die \begriff{Galoisgruppe} von $L\mid K$.
\end{definition}

\begin{remark}
	\proplbl{2_1_11}
	Ist $L\mid K$ endlich und galoissch, so gilt nach \propref{2_1_8} \begin{align*}
		\#\Gal(L\mid K) = [L:K].
	\end{align*}
\end{remark}

\begin{definition}
	\proplbl{2_1_12}
	Sei $G\le \Aut(L\mid K)$ Untergruppe. Dann ist\begin{align*}
		L^{\mathrm G} := \big\lbrace \alpha\in L\;\big|\; \forall \sigma\in G\colon\, \alpha^\sigma = \alpha\big\rbrace
	\end{align*}
	der \begriff{Fixkörper} von $G$.
\end{definition}

\begin{remark}
	\proplbl{2_1_13}
	$L^{\mathrm G}$ ist ein Teilkörper von $L$.
\end{remark}

\begin{proposition}[Artin]
	\proplbl{2_1_14}
	Sei $G\le \Aut(L)$ endlich, so ist $L\mid L^{\mathrm G}$ galoissch und $\Gal(L\mid L^{\mathrm G}) = G$.
\end{proposition}

\begin{proof}
	Sei $\alpha\in K = L^{\mathrm G}$. Dann ist \begin{align*}
		G_\alpha = \big\lbrace \sigma\in G\;\big|\; \alpha^\sigma = \alpha\big\rbrace \le G
	\end{align*}
	und die $G_\alpha$ partitionieren $G$:\begin{align*}
		G = \bigcup_{i=1}^m G_\alpha \sigma_i,
	\end{align*}
	wobei $m = [G:G_\alpha]$ und $\sigma_i$ ein Repräsentantensystem ist.
	
	Betrachte \begin{align*}
		f(X) = \prod_{i=1}^m ( X - \alpha^{\sigma_i}) \in L[X].
	\end{align*}
	Dannn gilt \begin{itemize}
		\item $f$ ist unabhängig von der Wahl der $\sigma_i$,
		\item $f(\alpha) = 0$,
		\item $f$ ist separabel, da $\alpha^{\sigma_i} = \alpha^{\sigma_j}$ $\Rightarrow$ $G_\alpha\sigma_i = G_\alpha \sigma_j$ $\Rightarrow$ $i=j$,
		\item $f\in K[X]$, da\begin{align*}
			\forall \tau\in G\colon\; G = G_\tau = \bigcup_{i=1}^m G_\alpha \sigma_i^\tau
		\end{align*}
		und \begin{align*}
			f^\tau(X) = \prod_{i=1}^m (X - \alpha^{\sigma_i\tau}) = f(X).
		\end{align*}
	\end{itemize}
	$L\mid L^{\mathrm G}$ ist also nach \propref{1_7_2} separabel.
	
	Für jedes $\alpha\in L$ gilt, dass $\deg(\alpha\mid K)\le \# G$ und \begin{align*}
		[L:K] \overset{\propref{1_9_4}}{\le} \# G \le \# \Aut(L\mid K) \overset{\propref{2_1_6}}{\le} [L:K].
	\end{align*}
	Daher: $G = \Aut(L\mid K)$ und aus\begin{align*}
		\#\Aut(L\mid K) = [L:K]
	\end{align*}
	folgt, dass die Erweiterung $L\mid K$ galoissch ist.
\end{proof}

\begin{conclusion}
	\proplbl{2_1_15}
	Sei $L\mid K$ endlich. Es gilt \begin{align*}
		L\mid K\;\text{galoisch}\quad\Leftrightarrow\quad L^{\Aut(L\mid K)} = K
	\end{align*}
\end{conclusion}

\begin{proof}
	\leavevmode
	\begin{itemize}[topsep=-6pt,widest={($\Rightarrow$)},leftmargin=*]
		\item[($\Rightarrow$)] $L\mid K$ galoissch \begin{itemize}[topsep=0pt,label={$\Rightarrow$}]
			\item[$\xRightarrow{\propref{2_1_8}}$] $\Aut(L\mid K) = [L:K]$
		\end{itemize}
		$K\subset L^{\Aut(L\mid K)} \subset L$ \begin{itemize}[topsep=0pt,label={$\Rightarrow$}]
			\item $[L:L^{\Aut(L\mid K)}] \overset{\propref{2_1_14}}{=} \# \Aut(L\mid K)$
			\item $K = L^{\Aut(L\mid K)}$
		\end{itemize}
	
		\item[($\Leftarrow$)] $L\mid K$ endlich \begin{itemize}[topsep=0pt]
			\item[$\xRightarrow{\propref{2_1_5}}$] $\Aut(L\mid K)$ endlich
			\item[$\xRightarrow{\propref{2_1_14}}$] $L\mid L^{\Aut(L\mid K)} = K$ ist galoissch
		\end{itemize}
	\end{itemize}
\end{proof}

\begin{lemma}
	\proplbl{2_1_16}
	Sind $K\subset L\subset M\subset \bar K$ Körper mit $L\mid K$ und $M\mid K$ normal, dann ist \begin{align*}
		\res_{M\mid L}\colon\;\left\lbrace\begin{array}{@{}c@{\;}c@{\;}c}
			\Aut(M\mid K) & \rightarrow & \Aut (L\mid K) \\
			\sigma & \mapsto & \sigma|_{L}
		\end{array}\right.
	\end{align*}
	ein Epimorphismus.
\end{lemma}

\begin{proof}
	Nach \propref{1_4_11} sind $\res_{\bar K\mid M}$ und $\res_{\bar K\mid L}$ surjektiv. Dann \begin{itemize}%[topsep=-\baselineskip]
		\item $\sigma|_{L}\in\Aut(L\mid K)$: Schreibe $\sigma = \res_{\bar K\mid M}(\tilde\sigma)$. Es gilt wegen \propref{2_1_2_4}: $\sigma(L) = \tilde\sigma(L) = L$.
		\item $\res_{M\mid L}$ ist Homomorphismus: klar
		\item $\res_{M\mid L}$ ist surjektiv: $\res_{\bar K\mid L} = \res_{M\mid L}\circ \res_{\bar K\mid M}$. Als zweiter Teil einer surjektiven Verkettung ist dieser selbst surjektiv.
	\end{itemize}
\end{proof}
\section{Der Hauptsatz der Galoistheorie}

$L\mid K$ ist endliche Galoiserweiterung mit $G = \Gal(L\mid K)$.

\begin{definition}
	Es sind \begin{itemize}
		\item $\zwk(L\mid K) = \lbrace F\mid K\subset F\subset L,\;F\,\text{Zwischenkörper}\rbrace$ die Menge der Zwischenkörper und
		\item $\ugr(G) = \lbrace H\mid H\le G\rbrace$ die Menge der Untergruppen.
	\end{itemize}
\end{definition}

\begin{theorem}[Galoiskorrespondenz]
	\proplbl{2_2_2}
	Es sind \begin{align*}
		&\left\lbrace\begin{array}{@{}c@{\;}c@{\;}c}
			\zwk(L\mid K) & \rightarrow & \ugr(G) \\
			F & \mapsto & F^\circ := \Gal(L\mid F)
		\end{array}\right.&
		&\left\lbrace \begin{array}{@{}c@{\;}c@{\;}c}
			\ugr(G) & \rightarrow & \zwk(L\mid K) \\
			H & \mapsto & H^\circ := L^H
		\end{array}\right.
	\end{align*}
	zueinander inverse Bijektionen. Weiterhin gilt für $F$, $F_1$, $F_2\in\zwk(L\mid K)$ mit $H = F^\circ$, $H_1 = F_1^\circ$ und $H_2 = F_2^\circ$ \begin{enumerate}[label={\roman*)}]
		\item \label{2_2_2_1} die Bijketion ist antiton \begin{align*}
			F_1\subset F_2 \quad\Leftrightarrow\quad H_1\supset H_2
		\end{align*}
		\item \label{2_2_2_2} die Bijektion ist indextreu, d.h.\begin{align*}
			[F_2:F_1] = (H_1:H_2),\quad\text{wenn}\;F_1\subset F_2
		\end{align*}
		\item \label{2_2_2_3} die Bijektion vertauscht Erzeugnis und Durchschnitt \begin{align*}
			(F_1\cap F_2)^\circ = \langle H_1,H_2\rangle\quad\text{und}\quad(F_1F_2)^\circ = H_1\cap H_2
		\end{align*}
		\item \label{2_2_2_4} die Bijektion ist mit Konjugation verträglich: $\forall \sigma\in G$ \begin{align*}
			\left( F^\sigma\right)^\circ = \left( F^\circ\right)^\sigma
		\end{align*}
		\item die Bijektion erhält Normalität: \begin{align*}
			F\mid K\;\text{normal}\quad\Leftrightarrow\quad H\unlhd G
		\end{align*}
		In disem Fall gilt: \begin{align*}
			\Gal(F\mid K) \cong \lnkset{G}{H} = \lnkset{\Gal(L\mid K)}{\Gal(L\mid F)}
		\end{align*}
	\end{enumerate}
\end{theorem}

\begin{proof}
	\leavevmode
	\begin{itemize}[topsep=-6pt]
		\item $F\in\zwk(L\mid K)$ $\xRightarrow{\propref{2_1_9}}$ $F$ galoissch $\xRightarrow{\propref{2_1_15}}$ $(F^\circ)^\circ = L^{F^\circ} = F$
		\item $H\in\ugr(G)$ $\xRightarrow{\propref{2_1_14}}$ $L\mid H^\circ$ galoissch mit $(H^\circ)^\circ = \Gal(L\mid H^\circ) = H$
	\end{itemize}
	\begin{enumerate}[label={\roman*)}]
		\item \begin{itemize}[]
			\item[$(\Leftarrow)$] klar, da $F_1 = H_1^\circ$, $F_2 = H_2^\circ$
			\item[($\Rightarrow$)] klarer
		\end{itemize}
		\item $L\mid F_i$ ist galoissch, daher folgt aus \propref{2_1_11} $[L:F_i] = \# H_i$ für $i=1$,$2$ und \begin{align*}
			[F_2:F_1] = \frac{[L:F_1]}{[L:F_2]} = \frac{\#H_1}{\#H_2} = (H_1 : H_2)
		\end{align*}
		\item \begin{itemize}[left=0pt]
			\item $F_1\cap F_2 \subset F_1 F_2$ $\Rightarrow$ $(F_1\cap F_2)^\circ \overset{\text{\hyperref[2_2_2_1]{i)}}}{\supset} \langle H_1,H_2\rangle$,
			
			$H_1$, $H_2\subset \langle H_1,H_2\rangle$ $\Rightarrow$ $F_1\cap F_2 \supset (\langle H_1,H_2\rangle)^\circ$ $\Rightarrow$ $(F_1\cap F_2)^\circ \subset \big( (\langle H_1,H_2\rangle)^\circ\big)^\circ = \langle H_1,H_2\rangle$
			\item $F_1$, $F_2\subset F_1F_2$ $\Rightarrow$ $H_1\cap H_2 \supset (F_1F_2)^\circ$
			
			$H_1\cap H_2\subset H_1$, $H_2$ $\Rightarrow$ $(H_1\cap H_2)^\circ \supset F_1F_2$ $\Rightarrow$ $(F_1F_2)^\circ \supset \big((H_1\cap H_2)^\circ\big)^\circ = H_1\cap H_2$
		\end{itemize}
		\item $(F^\sigma)^\circ = \lbrace \tau\in G\mid \tau|_{F^\sigma} = \id \rbrace = \lbrace \tau\in G\mid \tau(x) = x\;\forall x\in F^\sigma\rbrace = \lbrace \tau\in G\mid \tau(x^\sigma) = x^\sigma\;\forall x\in F\rbrace = \lbrace \tau\in G\mid \tau^{\sigma^{-1}}\in F^\circ\rbrace = (F^\circ)^\sigma$
			
		\item $F\mid K$ normal \begin{itemize}[topsep=\dimexpr-\baselineskip+2\lineskip\relax,widest={$\xLeftrightarrow{\propref{2_1_16}}$},leftmargin=*]
			\item[$\xLeftrightarrow{\propref{2_1_2}}$] $F^\sigma = F$ $\forall \sigma\in\Aut(\bar K\mid K)$
			\item[$\xLeftrightarrow{\propref{2_1_16}}$] $F^\sigma = F$ $\forall \sigma \in G$
			\item[$\xLeftrightarrow{\text{\hyperref[2_2_2_4]{iv)}}}$] $H^\sigma = H$ $\forall \sigma\in G$
			\item[$\Leftrightarrow$] $H\unlhd G$
			
			Sei $F\mid K$ normal. Nach \propref{2_1_16} gilt \begin{align*}
				\res\colon\;\left\lbrace\begin{array}{@{}c@{\;}c@{\;}c}
					\Gal(L\mid K) & \rightarrow & \Gal(F\mid K) \\
					\sigma & \mapsto & \sigma|_F
				\end{array}\right.
			\end{align*}
			ist ein Epimorphismus. \begin{itemize}[topsep=0pt]
				\item[$\Rightarrow$] $\Gal(F\mid K)\cong \Im(\res) \cong \lnkset{\Gal(L\mid K)}{\ker(\res)} \cong \lnkset{\Gal(L\mid K)}{\Gal(L\mid F)} = \lnkset{G}{H}$
			\end{itemize}
		\end{itemize}
	\end{enumerate}
\end{proof}

\begin{example}
	Betrachte $\mathbb Q(\sqrt2,\sqrt3)\mid\mathbb Q$:
	\begin{center}
		\includegraphics[width=1\linewidth]{tikz/splitting_a}
	\end{center}
	mit den Automorphismen \begin{align*}
		\sigma_1&\colon\;
			\begin{array}[t]{rcr}
			\sqrt 2 &\rightarrow& \sqrt 2\\
			\sqrt 3 &\rightarrow& \sqrt 3
			\end{array},&
		\sigma_2&\colon\;\begin{array}[t]{rcr}
			\sqrt 2&\rightarrow& \sqrt 2\\
			\sqrt 3&\rightarrow& -\sqrt 3
			\end{array},&
		\sigma_3&\colon\;\begin{array}[t]{rcr}
				\sqrt 2 &\rightarrow & -\sqrt 2\\
				\sqrt 3 &\rightarrow & \sqrt 3
			\end{array},&
		\sigma_4&\colon\;\begin{array}[t]{rcr}
			\sqrt 2 & \rightarrow & -\sqrt 2\\
			\sqrt 3 &\rightarrow & -\sqrt 3
		\end{array}.
	\end{align*}
\end{example}

\begin{remark}
	\proplbl{2_2_3}
	\begin{enumerate}[label={(\alph*)},topsep=0pt]
		\item Die Bijektivität in \propref{2_2_2} lässt sich mit \ref{2_2_2_1} und \ref{2_2_2_3} auch so ausdrücken: Das Bilden von Fixkörpern ist ein Verbandisomorphismus zwischen $\ugr(G)$ und $\zwk(L\mid K)$.
		\item Die Bijektivität gilt nicht für unendliche Galoiserweiterungen (siehe Übung)
		\item Mit \propref{2_2_2} erhalten wir einen neuen Beweis der Aussage \begin{align*}
			L\mid K\;\text{endlich galoissch}\quad\Leftrightarrow\quad \zwk(L\mid K)\;\text{endlich},
		\end{align*}
		was schon aus \propref{1_9_2} folgt.
	\end{enumerate}
\end{remark}

\begin{proposition}
	\proplbl{2_2_4}
	Sei $f\in K[X]$ separabel mit Nullstellen $\alpha_1$, $\dots$, $\alpha_n\in\bar K$ und sei $L=K(\alpha_1,\dots,\alpha_n)$ der \linebreak Zer\-fäl\-lungs\-kör\-per von $f$. Dann wirkt $G=\Gal(L\mid K)$  treu auf $X:= \lbrace\alpha_1,\dots,\alpha_n\rbrace$; der Homomorphismus \begin{align*}
		G\rightarrow \mathrm{Sym}(X) \cong S_n
	\end{align*}
	ist also eine Einbettung. Die Wirkung von $G$ auf $X$ ist genau dann transitiv, wenn $f$ irreduzibel ist.
\end{proposition}

\begin{proof}
	Sei $\sigma\in G$. \begin{itemize}[topsep=-6pt]
		\item treu: $\alpha_i^\sigma = \alpha_i$ $\forall i$ $\Rightarrow$ $\sigma=\id$, denn $L=K(\alpha_1,\dots,\alpha_n)$ und $\sigma|_K = \id_K$.
		\item Einbettung: GEO I.6.8
		\item \begin{tabular}[t]{@{}r@{$\quad\Leftrightarrow\quad$}l}
			transitiv & $\forall i$, $j$: $\exists \sigma\in G$: $\sigma(\alpha_i) = \alpha_j$ \\
			& $\forall i$, $j$: $\exists \sigma\in \Aut(\bar K\mid K)$: $\sigma(\alpha_i)= \alpha_j$ \\
			& $\alpha_1$, $\dots$, $\alpha_n$ sind paarweise $k$-konjugiert \\
			& $f = c\cdot \MinPol(\alpha_1\mid K)$, $c\in K^\times$ \\
			& $f$ irreduzibel
		\end{tabular}
	\end{itemize}
\end{proof}

\begin{definition}
	In der Situation von \propref{2_2_4} heißt \begin{align*}
		\Gal(f\mid K) &:= \Image (G\to\Sym(\alpha_1,\dots,\alpha_n))
	\end{align*}
	die Galoisgruppe von $f$. Man nennt $f$ galoissch, wenn $f$ irreduzibel ist und ein Wurzelkörper von $f$ schon ein Zerfällungskörper von $f$ ist.
\end{definition}

\begin{remark}
	\proplbl{2_2_6}
	\begin{enumerate}[topsep=0pt,label={(\alph*)}]
		\item Ist $L$ ein Zerfällungskörper von $f=\prod_{i=1}^n (X-\alpha_i)\in K[X]$, so gilt also \begin{align*}
			\Gal(L\mid K) \cong \Gal(f\mid K) \le \Sym(\lbrace \alpha_1,\dots,\alpha_n\rbrace) \cong S_n.
		\end{align*}
		\item Genau dann ist $f$ galoissch, wenn $G:= \Gal(f\mid K) \le S_n$ transitiv und $\# G = n$.
	\end{enumerate}
\end{remark}

\begin{example}
	Sei $f=X^3-2\in\mathbb Q[X]$. Die Nullstellen von $f$ sind \begin{align*}
		\alpha_1 &= \sqrt[3]{2},\\
		\alpha_2 &= \sqrt[3]{2}\zeta_3,\\
		\alpha_3 &= \sqrt[3]{2}\zeta_3^2.
	\end{align*}
	Der Zerfällungskörper ist $L:= \mathbb Q(\alpha_1,\alpha_2,\alpha_3) = \mathbb Q(\alpha_1,\alpha_2) = \mathbb Q(\sqrt[3]{2},\zeta_3)$. Weiterhin ist \begin{align*}
		G = \Gal(f\mid \mathbb Q) \le S_3;\quad\# G = [L:K] = 6\quad\Rightarrow\quad \Gal(L\mid K)\cong \Gal(f\mid K) = S_3 = \langle (1\,2\,3),(2\,3)\rangle
	\end{align*}
	Sei $\sigma\in G$ $\leftrightsquigarrow$ $(1\,2\,3),\;\text{also}\; (\sqrt[3]2)^\sigma = \zeta_3\sqrt[3]2,\; (\zeta_3\sqrt[3]2)^\sigma = \zeta_3^2\sqrt[3]2,\;(\zeta_3^2\sqrt[3]2)^\sigma = \sqrt[3]2$, $\zeta_3^\sigma = (\frac{\alpha_2}{\alpha_1})^\sigma = \zeta_3$.
	
	$G \ni \tau$ $\leftrightsquigarrow$ $(2\,3)$, also $(\sqrt[3]2)^\tau = \sqrt[3]2$, $(\zeta_3\sqrt[3]2)^\tau = \zeta_3^2\sqrt[3]2$, $(\zeta_3^2\sqrt[3]2)^\tau = \zeta_3 \sqrt[3]2$, $\zeta_3^\tau = (\frac{\alpha_2}{\alpha_1})^\tau = \zeta_3^2 = \zeta_3^{-1} = \bar{\zeta_3}$.
	
	\begin{tikzcd}[column sep=1em]
		& S_3 \arrow[ld, "3"'] \arrow[d, "3"'] \arrow[rd, "3"'] \arrow[rrd, "2"] &                                                                          &                                           &                                                  & G \arrow[ld, "3"'] \arrow[d, "3"'] \arrow[rd, "3"'] \arrow[rrd, "2"]                              &                                                  &                                       \\
		{\langle (1,2)\rangle} \arrow[rd, "2"']                                    & {\langle(2,3)\rangle} \arrow[d, "2"']                                  & {\langle(1,3)\rangle)} \arrow[ld, "2"']                                  & A_4 \arrow[lld, "3"]                      & \langle\tau\sigma^2\rangle \arrow[rd, "2"']      & \langle\tau\rangle \arrow[d, "2"']                                                                & \langle\tau\sigma\rangle \arrow[ld, "2"']        & \langle\sigma\rangle \arrow[lld, "3"] \\
		& 1                                                                      &                                                                          &  \arrow[dd, "\text{\propref{2_2_2}, \ref{2_2_2_1}, \ref{2_2_2_2}, \ref{2_2_2_4}}"]            &                                                  & \lbrace\id\rbrace                                                                                 &                                                  &                                       \\
		&                                                                        &                                                                          &                                           &                                                  &                                                                                                   &                                                  &                                       \\
		& L \arrow[ld, "2"'] \arrow[d, "2"'] \arrow[rd, "2"'] \arrow[rrd, "3"]   &                                                                          & \                                         &                                                  & {\mathbb Q(\sqrt[3]2,\zeta_3)} \arrow[ld, "2"'] \arrow[d, "2"'] \arrow[rd, "2"'] \arrow[rrd, "3"] &                                                  &                                       \\
		\Big(L^{\langle\tau\rangle}\Big)^{\langle\sigma^2\rangle} \arrow[rd, "3"'] & L^{\langle\tau\rangle} \arrow[d, "3"']                                 & \Big(L^{\langle\tau\rangle}\Big)^{\langle\sigma\rangle} \arrow[ld, "3"'] & L^{\langle\sigma\rangle} \arrow[lld, "2"] & {\mathbb Q(\zeta_3^2\sqrt[3]2)} \arrow[rd, "3"'] & {\mathbb Q(\sqrt[3]2)} \arrow[d, "3"']                                                            & {\mathbb Q(\zeta_3\sqrt[3]{2})} \arrow[ld, "3"'] & \mathbb Q(\zeta_3) \arrow[lld, "2"]   \\
		& \mathbb Q                                                              &                                                                          &                                           &                                                  & \mathbb Q                                                                                         &                                                  &                                      
	\end{tikzcd}
\end{example}

\part*{Anhang}
\addcontentsline{toc}{part}{Anhang}
\appendix

%\printglossary[type=\acronymtype]

\printindex

\end{document}
