\section{Der Hauptsatz der Galoistheorie}

$L\mid K$ ist endliche Galoiserweiterung mit $G = \Gal(L\mid K)$.

\begin{definition}
	Es sind \begin{itemize}
		\item $\zwk(L\mid K) = \lbrace F\mid K\subset F\subset L,\;F\,\text{Zwischenkörper}\rbrace$ die Menge der Zwischenkörper und
		\item $\ugr(G) = \lbrace H\mid H\le G\rbrace$ die Menge der Untergruppen.
	\end{itemize}
\end{definition}

\begin{theorem}[Galoiskorrespondenz]
	\proplbl{2_2_2}
	Es sind \begin{align*}
		&\left\lbrace\begin{array}{@{}c@{\;}c@{\;}c}
			\zwk(L\mid K) & \rightarrow & \ugr(G) \\
			F & \mapsto & F^\circ := \Gal(L\mid F)
		\end{array}\right.&
		&\left\lbrace \begin{array}{@{}c@{\;}c@{\;}c}
			\ugr(G) & \rightarrow & \zwk(L\mid K) \\
			H & \mapsto & H^\circ := L^H
		\end{array}\right.
	\end{align*}
	zueinander inverse Bijektionen. Weiterhin gilt für $F$, $F_1$, $F_2\in\zwk(L\mid K)$ mit $H = F^\circ$, $H_1 = F_1^\circ$ und $H_2 = F_2^\circ$ \begin{enumerate}[label={\roman*)}]
		\item \label{2_2_2_1} die Bijketion ist antiton \begin{align*}
			F_1\subset F_2 \quad\Leftrightarrow\quad H_1\supset H_2
		\end{align*}
		\item \label{2_2_2_2} die Bijektion ist indextreu, d.h.\begin{align*}
			[F_2:F_1] = (H_1:H_2),\quad\text{wenn}\;F_1\subset F_2
		\end{align*}
		\item \label{2_2_2_3} die Bijektion vertauscht Erzeugnis und Durchschnitt \begin{align*}
			(F_1\cap F_2)^\circ = \langle H_1,H_2\rangle\quad\text{und}\quad(F_1F_2)^\circ = H_1\cap H_2
		\end{align*}
		\item \label{2_2_2_4} die Bijektion ist mit Konjugation verträglich: $\forall \sigma\in G$ \begin{align*}
			\left( F^\sigma\right)^\circ = \left( F^\circ\right)^\sigma
		\end{align*}
		\item die Bijektion erhält Normalität: \begin{align*}
			F\mid K\;\text{normal}\quad\Leftrightarrow\quad H\unlhd G
		\end{align*}
		In disem Fall gilt: \begin{align*}
			\Gal(F\mid K) \cong \lnkset{G}{H} = \lnkset{\Gal(L\mid K)}{\Gal(L\mid F)}
		\end{align*}
	\end{enumerate}
\end{theorem}

\begin{proof}
	\leavevmode
	\begin{itemize}[topsep=-6pt]
		\item $F\in\zwk(L\mid K)$ $\xRightarrow{\propref{2_1_9}}$ $F$ galoissch $\xRightarrow{\propref{2_1_15}}$ $(F^\circ)^\circ = L^{F^\circ} = F$
		\item $H\in\ugr(G)$ $\xRightarrow{\propref{2_1_14}}$ $L\mid H^\circ$ galoissch mit $(H^\circ)^\circ = \Gal(L\mid H^\circ) = H$
	\end{itemize}
	\begin{enumerate}[label={\roman*)}]
		\item \begin{itemize}[]
			\item[$(\Leftarrow)$] klar, da $F_1 = H_1^\circ$, $F_2 = H_2^\circ$
			\item[($\Rightarrow$)] klarer
		\end{itemize}
		\item $L\mid F_i$ ist galoissch, daher folgt aus \propref{2_1_11} $[L:F_i] = \# H_i$ für $i=1$,$2$ und \begin{align*}
			[F_2:F_1] = \frac{[L:F_1]}{[L:F_2]} = \frac{\#H_1}{\#H_2} = (H_1 : H_2)
		\end{align*}
		\item \begin{itemize}[left=0pt]
			\item $F_1\cap F_2 \subset F_1 F_2$ $\Rightarrow$ $(F_1\cap F_2)^\circ \overset{\text{\hyperref[2_2_2_1]{i)}}}{\supset} \langle H_1,H_2\rangle$,
			
			$H_1$, $H_2\subset \langle H_1,H_2\rangle$ $\Rightarrow$ $F_1\cap F_2 \supset (\langle H_1,H_2\rangle)^\circ$ $\Rightarrow$ $(F_1\cap F_2)^\circ \subset \big( (\langle H_1,H_2\rangle)^\circ\big)^\circ = \langle H_1,H_2\rangle$
			\item $F_1$, $F_2\subset F_1F_2$ $\Rightarrow$ $H_1\cap H_2 \supset (F_1F_2)^\circ$
			
			$H_1\cap H_2\subset H_1$, $H_2$ $\Rightarrow$ $(H_1\cap H_2)^\circ \supset F_1F_2$ $\Rightarrow$ $(F_1F_2)^\circ \supset \big((H_1\cap H_2)^\circ\big)^\circ = H_1\cap H_2$
		\end{itemize}
		\item $(F^\sigma)^\circ = \lbrace \tau\in G\mid \tau|_{F^\sigma} = \id \rbrace = \lbrace \tau\in G\mid \tau(x) = x\;\forall x\in F^\sigma\rbrace = \lbrace \tau\in G\mid \tau(x^\sigma) = x^\sigma\;\forall x\in F\rbrace = \lbrace \tau\in G\mid \tau^{\sigma^{-1}}\in F^\circ\rbrace = (F^\circ)^\sigma$
			
		\item $F\mid K$ normal \begin{itemize}[topsep=\dimexpr-\baselineskip+2\lineskip\relax,widest={$\xLeftrightarrow{\propref{2_1_16}}$},leftmargin=*]
			\item[$\xLeftrightarrow{\propref{2_1_2}}$] $F^\sigma = F$ $\forall \sigma\in\Aut(\bar K\mid K)$
			\item[$\xLeftrightarrow{\propref{2_1_16}}$] $F^\sigma = F$ $\forall \sigma \in G$
			\item[$\xLeftrightarrow{\text{\hyperref[2_2_2_4]{iv)}}}$] $H^\sigma = H$ $\forall \sigma\in G$
			\item[$\Leftrightarrow$] $H\unlhd G$
			
			Sei $F\mid K$ normal. Nach \propref{2_1_16} gilt \begin{align*}
				\res\colon\;\left\lbrace\begin{array}{@{}c@{\;}c@{\;}c}
					\Gal(L\mid K) & \rightarrow & \Gal(F\mid K) \\
					\sigma & \mapsto & \sigma|_F
				\end{array}\right.
			\end{align*}
			ist ein Epimorphismus. \begin{itemize}[topsep=0pt]
				\item[$\Rightarrow$] $\Gal(F\mid K)\cong \Im(\res) \cong \lnkset{\Gal(L\mid K)}{\ker(\res)} \cong \lnkset{\Gal(L\mid K)}{\Gal(L\mid F)} = \lnkset{G}{H}$
			\end{itemize}
		\end{itemize}
	\end{enumerate}
\end{proof}

\begin{example}
	Betrachte $\mathbb Q(\sqrt2,\sqrt3)\mid\mathbb Q$:
	\begin{center}
		\includegraphics[width=1\linewidth]{tikz/splitting_a}
	\end{center}
	mit den Automorphismen \begin{align*}
		\sigma_1&\colon\;
			\begin{array}[t]{rcr}
			\sqrt 2 &\rightarrow& \sqrt 2\\
			\sqrt 3 &\rightarrow& \sqrt 3
			\end{array},&
		\sigma_2&\colon\;\begin{array}[t]{rcr}
			\sqrt 2&\rightarrow& \sqrt 2\\
			\sqrt 3&\rightarrow& -\sqrt 3
			\end{array},&
		\sigma_3&\colon\;\begin{array}[t]{rcr}
				\sqrt 2 &\rightarrow & -\sqrt 2\\
				\sqrt 3 &\rightarrow & \sqrt 3
			\end{array},&
		\sigma_4&\colon\;\begin{array}[t]{rcr}
			\sqrt 2 & \rightarrow & -\sqrt 2\\
			\sqrt 3 &\rightarrow & -\sqrt 3
		\end{array}.
	\end{align*}
\end{example}

\begin{remark}
	\proplbl{2_2_3}
	\begin{enumerate}[label={(\alph*)},topsep=0pt]
		\item Die Bijektivität in \propref{2_2_2} lässt sich mit \ref{2_2_2_1} und \ref{2_2_2_3} auch so ausdrücken: Das Bilden von Fixkörpern ist ein Verbandisomorphismus zwischen $\ugr(G)$ und $\zwk(L\mid K)$.
		\item Die Bijektivität gilt nicht für unendliche Galoiserweiterungen (siehe Übung)
		\item Mit \propref{2_2_2} erhalten wir einen neuen Beweis der Aussage \begin{align*}
			L\mid K\;\text{endlich galoissch}\quad\Leftrightarrow\quad \zwk(L\mid K)\;\text{endlich},
		\end{align*}
		was schon aus \propref{1_9_2} folgt.
	\end{enumerate}
\end{remark}

\begin{proposition}
	\proplbl{2_2_4}
	Sei $f\in K[X]$ separabel mit Nullstellen $\alpha_1$, $\dots$, $\alpha_n\in\bar K$ und sei $L=K(\alpha_1,\dots,\alpha_n)$ der \linebreak Zer\-fäl\-lungs\-kör\-per von $f$. Dann wirkt $G=\Gal(L\mid K)$  treu auf $X:= \lbrace\alpha_1,\dots,\alpha_n\rbrace$; der Homomorphismus \begin{align*}
		G\rightarrow \mathrm{Sym}(X) \cong S_n
	\end{align*}
	ist also eine Einbettung. Die Wirkung von $G$ auf $X$ ist genau dann transitiv, wenn $f$ irreduzibel ist.
\end{proposition}

\begin{proof}
	Sei $\sigma\in G$. \begin{itemize}[topsep=-6pt]
		\item treu: $\alpha_i^\sigma = \alpha_i$ $\forall i$ $\Rightarrow$ $\sigma=\id$, denn $L=K(\alpha_1,\dots,\alpha_n)$ und $\sigma|_K = \id_K$.
		\item Einbettung: GEO I.6.8
		\item \begin{tabular}[t]{@{}r@{$\quad\Leftrightarrow\quad$}l}
			transitiv & $\forall i$, $j$: $\exists \sigma\in G$: $\sigma(\alpha_i) = \alpha_j$ \\
			& $\forall i$, $j$: $\exists \sigma\in \Aut(\bar K\mid K)$: $\sigma(\alpha_i)= \alpha_j$ \\
			& $\alpha_1$, $\dots$, $\alpha_n$ sind paarweise $k$-konjugiert \\
			& $f = c\cdot \MinPol(\alpha_1\mid K)$, $c\in K^\times$ \\
			& $f$ irreduzibel
		\end{tabular}
	\end{itemize}
\end{proof}

\begin{definition}
	In der Situation von \propref{2_2_4} heißt \begin{align*}
		\Gal(f\mid K) &:= \Image (G\to\Sym(\alpha_1,\dots,\alpha_n))
	\end{align*}
	die Galoisgruppe von $f$. Man nennt $f$ galoissch, wenn $f$ irreduzibel ist und ein Wurzelkörper von $f$ schon ein Zerfällungskörper von $f$ ist.
\end{definition}

\begin{remark}
	\proplbl{2_2_6}
	\begin{enumerate}[topsep=0pt,label={(\alph*)}]
		\item Ist $L$ ein Zerfällungskörper von $f=\prod_{i=1}^n (X-\alpha_i)\in K[X]$, so gilt also \begin{align*}
			\Gal(L\mid K) \cong \Gal(f\mid K) \le \Sym(\lbrace \alpha_1,\dots,\alpha_n\rbrace) \cong S_n.
		\end{align*}
		\item Genau dann ist $f$ galoissch, wenn $G:= \Gal(f\mid K) \le S_n$ transitiv und $\# G = n$.
	\end{enumerate}
\end{remark}

\begin{example}
	Sei $f=X^3-2\in\mathbb Q[X]$. Die Nullstellen von $f$ sind \begin{align*}
		\alpha_1 &= \sqrt[3]{2},\\
		\alpha_2 &= \sqrt[3]{2}\zeta_3,\\
		\alpha_3 &= \sqrt[3]{2}\zeta_3^2.
	\end{align*}
	Der Zerfällungskörper ist $L:= \mathbb Q(\alpha_1,\alpha_2,\alpha_3) = \mathbb Q(\alpha_1,\alpha_2) = \mathbb Q(\sqrt[3]{2},\zeta_3)$. Weiterhin ist \begin{align*}
		G = \Gal(f\mid \mathbb Q) \le S_3;\quad\# G = [L:K] = 6\quad\Rightarrow\quad \Gal(L\mid K)\cong \Gal(f\mid K) = S_3 = \langle (1\,2\,3),(2\,3)\rangle
	\end{align*}
	Sei $\sigma\in G$ $\leftrightsquigarrow$ $(1\,2\,3),\;\text{also}\; (\sqrt[3]2)^\sigma = \zeta_3\sqrt[3]2,\; (\zeta_3\sqrt[3]2)^\sigma = \zeta_3^2\sqrt[3]2,\;(\zeta_3^2\sqrt[3]2)^\sigma = \sqrt[3]2$, $\zeta_3^\sigma = (\frac{\alpha_2}{\alpha_1})^\sigma = \zeta_3$.
	
	$G \ni \tau$ $\leftrightsquigarrow$ $(2\,3)$, also $(\sqrt[3]2)^\tau = \sqrt[3]2$, $(\zeta_3\sqrt[3]2)^\tau = \zeta_3^2\sqrt[3]2$, $(\zeta_3^2\sqrt[3]2)^\tau = \zeta_3 \sqrt[3]2$, $\zeta_3^\tau = (\frac{\alpha_2}{\alpha_1})^\tau = \zeta_3^2 = \zeta_3^{-1} = \bar{\zeta_3}$.
	
	\begin{tikzcd}[column sep=1em]
		& S_3 \arrow[ld, "3"'] \arrow[d, "3"'] \arrow[rd, "3"'] \arrow[rrd, "2"] &                                                                          &                                           &                                                  & G \arrow[ld, "3"'] \arrow[d, "3"'] \arrow[rd, "3"'] \arrow[rrd, "2"]                              &                                                  &                                       \\
		{\langle (1,2)\rangle} \arrow[rd, "2"']                                    & {\langle(2,3)\rangle} \arrow[d, "2"']                                  & {\langle(1,3)\rangle)} \arrow[ld, "2"']                                  & A_4 \arrow[lld, "3"]                      & \langle\tau\sigma^2\rangle \arrow[rd, "2"']      & \langle\tau\rangle \arrow[d, "2"']                                                                & \langle\tau\sigma\rangle \arrow[ld, "2"']        & \langle\sigma\rangle \arrow[lld, "3"] \\
		& 1                                                                      &                                                                          &  \arrow[dd, "\text{\propref{2_2_2}, \ref{2_2_2_1}, \ref{2_2_2_2}, \ref{2_2_2_4}}"]            &                                                  & \lbrace\id\rbrace                                                                                 &                                                  &                                       \\
		&                                                                        &                                                                          &                                           &                                                  &                                                                                                   &                                                  &                                       \\
		& L \arrow[ld, "2"'] \arrow[d, "2"'] \arrow[rd, "2"'] \arrow[rrd, "3"]   &                                                                          & \                                         &                                                  & {\mathbb Q(\sqrt[3]2,\zeta_3)} \arrow[ld, "2"'] \arrow[d, "2"'] \arrow[rd, "2"'] \arrow[rrd, "3"] &                                                  &                                       \\
		\Big(L^{\langle\tau\rangle}\Big)^{\langle\sigma^2\rangle} \arrow[rd, "3"'] & L^{\langle\tau\rangle} \arrow[d, "3"']                                 & \Big(L^{\langle\tau\rangle}\Big)^{\langle\sigma\rangle} \arrow[ld, "3"'] & L^{\langle\sigma\rangle} \arrow[lld, "2"] & {\mathbb Q(\zeta_3^2\sqrt[3]2)} \arrow[rd, "3"'] & {\mathbb Q(\sqrt[3]2)} \arrow[d, "3"']                                                            & {\mathbb Q(\zeta_3\sqrt[3]{2})} \arrow[ld, "3"'] & \mathbb Q(\zeta_3) \arrow[lld, "2"]   \\
		& \mathbb Q                                                              &                                                                          &                                           &                                                  & \mathbb Q                                                                                         &                                                  &                                      
	\end{tikzcd}
\end{example}