\section{Normale Körpererweiterungen}
Sei $K$ Körper, $\bar K$ ein fixierter algebraischer Abschluss von $K$ und $L$ ein Zwischenkörper $K\subseteq L\subseteq \bar K$.

\begin{definition}
	$L\mid K$ ist \begriff{Normal} :$\Leftrightarrow$ Ist $\alpha\in L$ und $\beta\in\bar K$ $K$-konjugiert, so ist $\beta\in L$.
\end{definition}

\begin{proposition}
	\proplbl{2_1_2}
	Ist $L\mid K$ endlich, so sind äquivalent \begin{propenum}
		\item $L\mid K$ ist normal
		\item Jedes irreduzible $f\in K[X]$, das eine Nullstelle in $L$ hat, zerfällt über $L$ in Linearfaktoren
		\item $L$ ist der Zerfällungskörper von $f\in K[X]$
		\item \proplbl{2_1_2_4} Für jedes $\sigma\in\Aut(\bar K\mid K)$ ist $\sigma(L) = L$
		\item Jedes $\sigma\in\Aut(\bar K\mid K)$ ist $\sigma(L)\subseteq L$
	\end{propenum}
\end{proposition}

\begin{proof}\leavevmode
	\begin{itemize}[widest={(1) $\Rightarrow$ (2)},leftmargin=*,topsep=-6pt]
		\item[(1) $\Rightarrow$ (2)] klar nach \propref{1_4_14}
		\item[(2) $\Rightarrow$ (3)] Sei $L = K(\alpha_1,\dots,\alpha_n)$. Mit \begin{equation*}
			f = \prod_{i=1}^n \MinPol(\alpha_i\mid K)
		\end{equation*}
		ist $L$ der Zerfällungskörper von $f$.
		\item[(3) $\Rightarrow$ (4)] Ist $f$ der Zerfällungskörper von\begin{equation*}
			f = \prop_{i=1}^n (X - X_i),
		\end{equation*}
		und $\sigma\in\Aut(\bar K\mid K)$, so permutiert $\sigma$ die Nullstellen $\lbrace \alpha_1,\dots,\alpha_n\rbrace$ von $f$, folglich \begin{equation*}
			\sigma(L) = \sigma\big( K(\alpha_1,\dots,\alpha_n)\big) = K\big(\sigma(\alpha_1),\dots,\sigma(\alpha_n)\big) = K(\alpha_1,\dots,\alpha_n) = L.
		\end{equation*}
		\item[(4) $\Rightarrow$ (5)] trivial
		\item[(5) $\Rightarrow$ (1)] trivial
	\end{itemize}
\end{proof}

\begin{example}
	\begin{enumerate}[label={\alph*)}]
		\item $K\mid K$ ist normal
		\item $\bar K\mid K$ ist normal
		\item $\bar K_{\mathrm S} \mid K$ ist normal (\propref{1_7_7})
		\item $[L:K] = 2$ $\Rightarrow$ $L\mid K$ ist normal
		
		($\deg(f) = 2$, $f$ hat Nullstelle $\Rightarrow$ $f$ zerfällt in Linearfaktoren)
		\item $L = \mathbb Q(\sqrt[3]2)$, $[L:\mathbb Q] = 3$ $L\mid Q$ ist nicht normal, die zu $\sqrt[3]2$ $\mathbb Q$-konjugierte Elemente $\zeta_3 \sqrt[3]2$ und $\zeta_3^2 \sqrt[3]2$ liegen \emph{nicht} in $L$ (\propref{1_3_11_b})
		\item $Sei \alpha = \sqrt[4]2\in\mathbb R_{\ge 0}$  und $f = \MinPol(\alpha\mid\mathbb Q) = X^4 - 2$. Dann sind die $\mathbb Q$-konjugierten $\pm \sqrt[4]2$ und $i\sqrt[4]2$. Da $i\sqrt[4]2\notin\mathbb R$ ist $\mathbb Q(\alpha)\mid \mathbb Q$ nicht normal und \begin{equation*}
			\underbrace{\mathbb Q(\sqrt[4]2) \;\, \underset{\text{normal}}{\overset{2}{\rule[0.1\baselineskip]{3em}{0.1pt}}} \;\, \mathbb  Q(\sqrt 2) \;\, \underset{\text{normal}}{\overset{2}{\rule[0.1\baselineskip]{3em}{0.1pt}}} \;\, \mathbb Q,}_{\text{nicht normal}}
		\end{equation*}
		also ist Normalität nicht transitiv.
	\end{enumerate}
\end{example}

\begin{conclusion}
	\proplbl{2_1_4}
	Sei $L\mid K$ endlich und seien $K\subseteq L_1$, $L_2\subseteq L$ Zwischenkörper. Dann \begin{enumerate}[label={(\alph*)}]
		\item Sind $L_1\mid K$ und $L_2\mid K$ normal, so auch $L_1\cap L_2\mid K$ und $L_1L_2 \mid K$
		\item Ist $L\mid K$ normal, so auch $L\mid L_1$
	\end{enumerate}
\end{conclusion}

\begin{proof}\leavevmode
\begin{enumerate}[label={\alph*)},topsep=-6pt]
	\item \begin{itemize}[left=0pt]
		\item $L1\cap L_2$: klar aus Definition
		\item $L_1L_2$: Sei $\sigma\in\Aut(\bar K\mid K)$ $\Rightarrow$ $\sigma(L_1L_2)  = \sigma(L_1)\sigma(L_2) = L_1 L_2$
	\end{itemize}
	\item klar, da $\Aut(\bar L_1\mid L_1)\subseteq \Aut(\bar K\mid K)$
\end{enumerate}
\end{proof}

\begin{proposition}
	\proplbl{2_1_5}
	Sei $L\mid K$ endlich. Es ist \begin{equation*}
		\# \Aut(L\mid K) \le [L:K]_{\mathrm S}
	\end{equation*}
	mit Gleichheit, wenn die Erweiterung normal ist.
\end{proposition}

\begin{proof}
	Es ist \begin{equation*}
		\Aut(L\mid K) = \Hom_K(L, L) = \big\lbrace \sigma\in\Hom_K(L,\bar K)\;\big|\; \sigma(L)\subseteq L\big\rbrace \subseteq \Hom_K(L,\bar K),
	\end{equation*}
	sodass $\# \Aut(L\mid K) \le \# \Hom_K(L,\bar K) = [L:K]_{\mathrm S}$.
	
	Es gilt: $\Aut(L\mid K) = \Hom_K(L\mid \bar K)$ \begin{itemize}[topsep=0pt,label={$\Leftrightarrow$},widest={<I.4.11>},leftmargin=*]
		\item $\forall \sigma\in \Hom_K(L\mid \bar K)$: $\sigma(L)\subseteq L$
		\item[$\xLeftrightarrow{\propref{1_4_11}}$] $\forall \sigma\in\Aut(\bar K\mid K)$: $\sigma(L)\subseteq L$
		\item[$\xLeftrightarrow{\propref{2_1_2}}$] $L\mid K$ normal.
	\end{itemize}
\end{proof}

\begin{remark}
	\proplbl{2_1_6}
	Es ist also \begin{equation*}
		\Aut(L\mid K) \overset{\circled{\tiny  1}}{\le} [L:K]_{\mathrm S} \overset{\circled{\tiny2}}{\le} [L:K],
	\end{equation*}
	wobei gilt: \begin{enumerate}[label=\protect\circled{\arabic*}]
		\item ist Gleichheit :$\xLeftrightarrow{\propref{2_1_5}}$ $L\mid K$ normal
		\item ist Gleichheit :$\xLeftrightarrow{\propref{1_7_6}}$ $L\mid K$ separabel
	\end{enumerate}
\end{remark}

\begin{definition}
	$L\mid K$ ist \begriff{galoissch} (oder Galoiserweiterung) $\Leftrightarrow$ $L\mid K$ ist normal und separabel
\end{definition}

\begin{proposition}
	\proplbl{2_1_8}
	Ist $L\mid K$ endlich, so sind äquivalent \begin{enumerate}[label={(\arabic*)}]
		\item $L\mid K$ ist galoissch
		\item Jedes $\alpha\in L$ hat $\deg(\alpha\mid L)$ viele $K$-konjugierte in $L$
		\item $L$ ist Zerfällungskörper eines irreduziblen, separablen Polynoms $f\in K[X]$
		\item $L$ ist Zerfällungskörper eines separablen Polynoms $f\in K[X]$
		\item $\#\Aut(L\mid K) = [L:K]$
	\end{enumerate}
\end{proposition}
\begin{proof}\leavevmode
	\begin{itemize}[topsep=-6pt,widest={(1) $\Leftrightarrow$ (3)},leftmargin=*]
		\item[(1) $\Leftrightarrow$ (5)] \propref{2_1_6}
		\item[(1) $\Leftrightarrow$ (2)] $L\mid K$ separabel $\Leftrightarrow$ jdes $\alpha\in L$ hat $\deg(\alpha\mid K)$ viele $K$-konjugierte in $\bar K$. \\
		$L\mid K$ normal $\Leftrightarrow$ alle $K$-konjugierte von $\alpha\in L$ liegen in $L$.
		\item[(1) $\Rightarrow$ (3)] $L\mid K$ separabel $\xRightarrow{\propref{1_9_4}}$ $L = K(\alpha)$ einfach.\\
		$L\mid K$ normal $\Rightarrow$ $L$ ist Zerfällungskörper von $\MinPol(\alpha\mid K)$
		\item[(3) $\Rightarrow$ (4)] trivial
		\item[(4) $\Rightarrow$ (1)] \propref{2_1_2} und \propref{1_7_6}
	\end{itemize}
\end{proof}

\begin{conclusion}
	\proplbl{2_1_9}
	Sei $L\mid K$ endlich und seien $K\subseteq L_1$, $L_2\subseteq L$ Zwischenkörper. \begin{enumerate}[label={(\alph*)}]
		\item Sind $L_1\mid K$ und $L_2\mid K$ galoissch, so auch $L_1\cap L_2\mid K$ und $L_1L_2 \mid K$
		\item Ist $L\mid K$ galoissch, so auch $L\mid L_1$
	\end{enumerate}
\end{conclusion}

\begin{proof}
	\propref{2_1_4}, \propref{1_7_8} und \propref{1_7_9}.
\end{proof}

\begin{definition}
	\proplbl{2_1_10}
	Ist $L\mid K$ galoissch, so heißt \begin{align*}
		\Aut(L\mid K) = \Gal(L\mid K)
	\end{align*}
	die \begriff{Galoisgruppe} von $L\mid K$.
\end{definition}

\begin{remark}
	\proplbl{2_1_11}
	Ist $L\mid K$ endlich und galoissch, so gilt nach \propref{2_1_8} \begin{align*}
		\#\Gal(L\mid K) = [L:K].
	\end{align*}
\end{remark}

\begin{definition}
	\proplbl{2_1_12}
	Sei $G\le \Aut(L\mid K)$ Untergruppe. Dann ist\begin{align*}
		L^{\mathrm G} := \big\lbrace \alpha\in L\;\big|\; \forall \sigma\in G\colon\, \alpha^\sigma = \alpha\big\rbrace
	\end{align*}
	der \begriff{Fixkörper} von $G$.
\end{definition}

\begin{remark}
	\proplbl{2_1_13}
	$L^{\mathrm G}$ ist ein Teilkörper von $L$.
\end{remark}

\begin{proposition}[Artin]
	\proplbl{2_1_14}
	Sei $G\le \Aut(L)$ endlich, so ist $L\mid L^{\mathrm G}$ galoissch und $\Gal(L\mid L^{\mathrm G}) = G$.
\end{proposition}

\begin{proof}
	Sei $\alpha\in K = L^{\mathrm G}$. Dann ist \begin{align*}
		G_\alpha = \big\lbrace \sigma\in G\;\big|\; \alpha^\sigma = \alpha\big\rbrace \le G
	\end{align*}
	und die $G_\alpha$ partitionieren $G$:\begin{align*}
		G = \bigcup_{i=1}^m G_\alpha \sigma_i,
	\end{align*}
	wobei $m = [G:G_\alpha]$ und $\sigma_i$ ein Repräsentantensystem ist.
	
	Betrachte \begin{align*}
		f(X) = \prod_{i=1}^m ( X - \alpha^{\sigma_i}) \in L[X].
	\end{align*}
	Dannn gilt \begin{itemize}
		\item $f$ ist unabhängig von der Wahl der $\sigma_i$,
		\item $f(\alpha) = 0$,
		\item $f$ ist separabel, da $\alpha^{\sigma_i} = \alpha^{\sigma_j}$ $\Rightarrow$ $G_\alpha\sigma_i = G_\alpha \sigma_j$ $\Rightarrow$ $i=j$,
		\item $f\in K[X]$, da\begin{align*}
			\forall \tau\in G\colon\; G = G_\tau = \bigcup_{i=1}^m G_\alpha \sigma_i^\tau
		\end{align*}
		und \begin{align*}
			f^\tau(X) = \prod_{i=1}^m (X - \alpha^{\sigma_i\tau}) = f(X).
		\end{align*}
	\end{itemize}
	$L\mid L^{\mathrm G}$ ist also nach \propref{1_7_2} separabel.
	
	Für jedes $\alpha\in L$ gilt, dass $\deg(\alpha\mid K)\le \# G$ und \begin{align*}
		[L:K] \overset{\propref{1_9_4}}{\le} \# G \le \# \Aut(L\mid K) \overset{\propref{2_1_6}}{\le} [L:K].
	\end{align*}
	Daher: $G = \Aut(L\mid K)$ und aus\begin{align*}
		\#\Aut(L\mid K) = [L:K]
	\end{align*}
	folgt, dass die Erweiterung $L\mid K$ galoissch ist.
\end{proof}

\begin{conclusion}
	\proplbl{2_1_15}
	Sei $L\mid K$ endlich. Es gilt \begin{align*}
		L\mid K\;\text{galoisch}\quad\Leftrightarrow\quad L^{\Aut(L\mid K)} = K
	\end{align*}
\end{conclusion}

\begin{proof}
	\leavevmode
	\begin{itemize}[topsep=-6pt,widest={($\Rightarrow$)},leftmargin=*]
		\item[($\Rightarrow$)] $L\mid K$ galoissch \begin{itemize}[topsep=0pt,label={$\Rightarrow$}]
			\item[$\xRightarrow{\propref{2_1_8}}$] $\Aut(L\mid K) = [L:K]$
		\end{itemize}
		$K\subset L^{\Aut(L\mid K)} \subset L$ \begin{itemize}[topsep=0pt,label={$\Rightarrow$}]
			\item $[L:L^{\Aut(L\mid K)}] \overset{\propref{2_1_14}}{=} \# \Aut(L\mid K)$
			\item $K = L^{\Aut(L\mid K)}$
		\end{itemize}
	
		\item[($\Leftarrow$)] $L\mid K$ endlich \begin{itemize}[topsep=0pt]
			\item[$\xRightarrow{\propref{2_1_5}}$] $\Aut(L\mid K)$ endlich
			\item[$\xRightarrow{\propref{2_1_14}}$] $L\mid L^{\Aut(L\mid K)} = K$ ist galoissch
		\end{itemize}
	\end{itemize}
\end{proof}

\begin{lemma}
	\proplbl{2_1_16}
	Sind $K\subset L\subset M\subset \bar K$ Körper mit $L\mid K$ und $M\mid K$ normal, dann ist \begin{align*}
		\res_{M\mid L}\colon\;\left\lbrace\begin{array}{@{}c@{\;}c@{\;}c}
			\Aut(M\mid K) & \rightarrow & \Aut (L\mid K) \\
			\sigma & \mapsto & \sigma|_{L}
		\end{array}\right.
	\end{align*}
	ein Epimorphismus.
\end{lemma}

\begin{proof}
	Nach \propref{1_4_11} sind $\res_{\bar K\mid M}$ und $\res_{\bar K\mid L}$ surjektiv. Dann \begin{itemize}%[topsep=-\baselineskip]
		\item $\sigma|_{L}\in\Aut(L\mid K)$: Schreibe $\sigma = \res_{\bar K\mid M}(\tilde\sigma)$. Es gilt wegen \propref{2_1_2_4}: $\sigma(L) = \tilde\sigma(L) = L$.
		\item $\res_{M\mid L}$ ist Homomorphismus: klar
		\item $\res_{M\mid L}$ ist surjektiv: $\res_{\bar K\mid L} = \res_{M\mid L}\circ \res_{\bar K\mid M}$. Als zweiter Teil einer surjektiven Verkettung ist dieser selbst surjektiv.
	\end{itemize}
\end{proof}